%-------------------------------------------------------------------------------
% User guide
%-------------------------------------------------------------------------------

% Custom environments and page setups to meet thesis requirements  for Western
% are in westerntemplate.tex and must be loaded in immediately after packages.

% Custom commands:
% \degree for degree symbol
% \u for mu symbol
% \tilde for tilde symbol
% \dummyref = Create a dummy reference link
% \dummyfig = Create a dummy figure link
% \dummytab = Create a dummy table link
% \dummysupfig = Create a dummy supplementary figure link
% \dummysuptab = Create a dummy supplementary table link

% Figures can be referenced in-text by:
% (Figure \ref{fig:macrolide-chemical-structures})

% To-do notes:
% To create a to-do note: \todo{This is a note.}
% To create a missing figure: \missingfigure{This text will appear inside the
% missing figure.}

% Supplementary figures and tables:
% Can use \begin{suppfigure} in place of \begin{figure}, and \begin{supptable}
% in place of \begin{table}
% Can print lists of supplementary figures and tables using \listofsuppfigures,
% \listofsupptables

%-------------------------------------------------------------------------------
% Packages
%-------------------------------------------------------------------------------

% Doc and page setup
% \documentclass[12pt, oneside]{report}
\documentclass[12pt, twoside]{report} % Uncomment for book binding margins

% Reference management
\usepackage[colorlinks = true,
            linkcolor = blue,
            urlcolor  = blue,
            citecolor = blue,
            anchorcolor = blue]{hyperref} % For cross-referenced hyperlinks, with colouring; comment out for hardcopy
\usepackage[style=authoryear, backend=biber, hyperref=true, natbib=true]{biblatex}

% Font
%\usepackage{mathptmx} % For pseudo-Times New Roman typeface when using serif
\usepackage{helvet} % For Helvetica typeface when using sans-serif
\renewcommand{\familydefault}{\sfdefault} % Uncomment for serif typeface

% Graphics
\usepackage{graphicx}
\usepackage{booktabs} % For horizontal bars to make 'science-y' looking tables
\usepackage{float} % Allows placement of figures exactly where inserted
\usepackage{newfloat} % Required for setting up supplementary figures and tables
\usepackage{array} % Required for centering columns in tables with defined width
\usepackage{longtable} % For creating tables that span multiple pages
\usepackage{enumitem} % For resuming enumerated lists
\usepackage[plain]{fancyref} % For advanced cross-referencing features
\usepackage{xcolor}
\usepackage{caption}
\usepackage{chngcntr} % Figure numbering
\usepackage{glossaries} % For abbreviations
\usepackage{textcomp} % For trademark symbols
\usepackage{todonotes} % For inserting to-do notes; can be used for commenting
\usepackage{marvosym} % For inserting the cross symbol % \Cross

% Western thesis template packages
\usepackage{tocloft} % For custom formatting of ToC and List of 'X'
\usepackage{appendix}
\usepackage{graphicx}
\usepackage{amsmath} % Can get rid of this if I'm not using any equations

% Line spacing
\usepackage{setspace}
\onehalfspacing % For 1.5 line spacing
%\doublespacing % For double line spacing

%-------------------------------------------------------------------------------
% Initialize
%-------------------------------------------------------------------------------

\newcommand{\degree}{$^\circ$} % Degree symbol
\renewcommand{\u}{$\mu$} % Mu symbol
\renewcommand{\tilde}{$\sim$} % Tilde symbol

% Create commands for dummy references
\newcommand{\dummyref}{(\textbf{\textcolor{red}{ref}})}
\newcommand{\dummyfig}{(Figure \textbf{\textcolor{red}{XX}})}
\newcommand{\dummytab}{(Table \textbf{\textcolor{red}{XX}})}
\newcommand{\dummysupfig}{(Supplementary Figure \textbf{\textcolor{red}{XX}})}
\newcommand{\dummysuptab}{(Supplementary Table \textbf{\textcolor{red}{XX}})}

% Figure and table captions
\floatstyle{plaintop} % For placing captions above tables
\restylefloat{table} % Place captions above tables
\definecolor{pastelred}{HTML}{D73231}
\captionsetup{labelfont = {bf, color = pastelred},
              font = {stretch = 1.25},
              justification = justified,
              labelsep = quad
              }
\counterwithout{figure}{chapter} % For continuous figure numbering
\counterwithout{table}{chapter} % For continuous table numbering

% Other
\DeclareLanguageMapping{english}{american-apa} % Required for proper citation using APA
\newcolumntype{P}[1]{>{\centering\arraybackslash}p{#1}} % For centering columns in tables (use P instead of p)
\addbibresource{references.bib}
\makeglossaries
\setuptodonotes{inline, color=green!20}
\DeclareFloatingEnvironment[name={Supplementary Figure},fileext=lsf,listname={List of Supplementary Figures}]{suppfigure} % Set up supplementary figures
\DeclareFloatingEnvironment[name={Supplementary Table},fileext=lsf,listname={List of Supplementary Tables}]{supptable} % Set up supplementary figures

% Avoid hyphenating words that run onto new lines when using text justification
\tolerance=1
\emergencystretch=\maxdimen
\hyphenpenalty=10000
\hbadness=10000

%-------------------------------------------------------------------------------
% Abbreviations
%-------------------------------------------------------------------------------

\newglossaryentry{ANOVA}{name={ANOVA}, description={Analysis of variance}}
\newglossaryentry{CARD}{name={CARD}, description={Comprehensive Antibiotic Resistance Database}}
\newglossaryentry{CLR}{name={CLR}, description={Center log ratio}}
\newglossaryentry{COG}{name={COG}, description={Cluster of Orthologous Groups}}
\newglossaryentry{Meta-MARC}{name={Meta-MARC}, description={Metagenomic Markov models for Antimicrobial Resistance Characterization}}
\newglossaryentry{PERMANOVA}{name={PERMANOVA}, description={Permutational multivarate analysis of variance}}
\newglossaryentry{PCA}{name={PCA}, description={Principal component analysis}}
\newglossaryentry{PNEC}{name={PNEC}, description={Predicted no-effect concentration}}

%-------------------------------------------------------------------------------
% Preliminary document
%-------------------------------------------------------------------------------

% Western thesis template requirements
%%%%%%%%%%%%%%%%%%%%%%%%%%%%%%%%%%%%%%%%%%%%%%%%%%%%%%%%%%%%%%%%%%%%%%%%
%%                                                                    %%
%%                    ***   I M P O R T A N T   ***                   %%
%%                                                                    %%
%% Fill in the following fields with the required information:        %%
%%  - \department{...}  name of the graduate department               %%
%%  - \degreename{...}      name of the degree obtained                   %%
%%  - \author{...}      name of the author                            %%
%%  - \title{...}       title of the thesis                           %%
%%  - \gyear{...}       year of graduation                            %%
%%  - \super{...}    supervisor
%%  - \firstname, \middlename, \lastname... there is additional documentation by the actual fields, so I'll leave it at that
%%%%%%%%%%%%%%%%%%%%%%%%%%%%%%%%%%%%%%%%%%%%%%%%%%%%%%%%%%%%%%%%%%%%%%%%

\makeatletter
\numberwithin{figure}{chapter}
\newenvironment{acknowledgements}%
{\clearemptydoublepage
 \begin{center}
  \section*{Acknowledgements}
 \end{center}
 \begingroup
}{\newpage\endgroup}

\newenvironment{dedication}%
{\clearemptydoublepage
 \begin{center}
  \section*{Dedication}
 \end{center}
 \begingroup
}{\newpage\endgroup}

\newenvironment{preliminary}%
{\pagestyle{plain}\pagenumbering{roman}}%
{\pagenumbering{arabic}}

%\addtoreflist{chapter}
\newtheorem{theorem}{Theorem}[section]
\newtheorem{lemma}[theorem]{Lemma}
\newtheorem{proposition}[theorem]{Proposition}
\newtheorem{corollary}[theorem]{Corollary}

\newenvironment{proof}[1][Proof]{\begin{trivlist}
\item[\hskip \labelsep {\bfseries #1}]}{\end{trivlist}}
\newenvironment{definition}[1][Definition]{\begin{trivlist}
\item[\hskip \labelsep {\bfseries #1}]}{\end{trivlist}}
\newenvironment{example}[1][Example]{\begin{trivlist}
\item[\hskip \labelsep {\bfseries #1}]}{\end{trivlist}}
\newenvironment{remark}[1][Remark]{\begin{trivlist}
\item[\hskip \labelsep {\bfseries #1}]}{\end{trivlist}}

\newcommand{\qed}{\nobreak \ifvmode \relax \else
      \ifdim\lastskip<1.5em \hskip-\lastskip
      \hskip1.5em plus0em minus0.5em \fi \nobreak
      \vrule height0.75em width0.5em depth0.25em\fi}

% Default values for title page.

%% To produce output with the desired line spacing, the argument of
%% \spacing should be multiplied by 5/6 = 0.8333, so that 1 1/2 spaced
%% corresponds to \spacing{1.5} and double spaced is \spacing{1.66}.
\def\normalspacing{1.25} % default line spacing


%% Define the "thesis" page style.
\if@twoside % If two-sided printing.
\def\ps@thesis{\let\@mkboth\markboth
   \def\@oddfoot{}
   \let\@evenfoot\@oddfoot
   \def\@oddhead{
      {\sc\rightmark} \hfil \rm\thepage
      }
   \def\@evenhead{
      \rm\thepage \hfil {\sc\leftmark}
      }
   \def\chaptermark##1{\markboth{\ifnum \c@secnumdepth >\m@ne
      Chapter\ \thechapter. \ \fi ##1}{}}
   \def\sectionmark##1{\markright{\ifnum \c@secnumdepth >\z@
      \thesection. \ \fi ##1}}}
\else % If one-sided printing.
\def\ps@thesis{\let\@mkboth\markboth
   \def\@oddfoot{}
   \def\@oddhead{
      {\sc\rightmark} \hfil \rm\thepage
      }
   \def\chaptermark##1{\markright{\ifnum \c@secnumdepth >\m@ne
      Chapter\ \thechapter. \ \fi ##1}}}
\fi

\pagestyle{thesis}
% Set up page layout.
\setlength{\textheight}{9in} % Height of the main body of the text
\setlength{\topmargin}{-.5in} % .5" margin on top of page
\setlength{\headsep}{.5in}  % space between header and top of body
\addtolength{\headsep}{-\headheight} % See The LaTeX Companion, p 85
\setlength{\footskip}{.5in}  % space between footer and bottom of body
\setlength{\textwidth}{6.25in} % width of the body of the text
\setlength{\oddsidemargin}{.25in} % 1.25" margin on the left for odd pages
\setlength{\evensidemargin}{0in} % 1.25"  margin on the right for even pages

% Marginal notes
\setlength{\marginparwidth}{.75in} % width of marginal notes
\setlength{\marginparsep}{.125in} % space between marginal notes and text

% Make each page fill up the entire page. comment this out if you
% prefer.
\flushbottom

\setcounter{tocdepth}{3} % Number the subsubsections
\def\normalspacing{1.25} % default line spacing

\newcommand\isco[1]{%
  \edef\@tempa{#1}%
  \def\@tempb{}%
  \ifx\@tempa\@tempb
	\else \\\underline{Co-Supervisor:}\vspace{0.35in}\\\dots\dots\dots\dots\dots\dots\dots\\{#1}\\
  \fi
}

\newcommand\isjoint[1]{%
  \edef\@tempa{#1}%
  \def\@tempb{}%
  \ifx\@tempa\@tempb
	\else \\\underline{Joint Supervisor:}\vspace{0.35in}\\\dots\dots\dots\dots\dots\dots\dots\\{#1}\\
  \fi
}

\newcommand\isalt[1]{%
  \edef\@tempa{#1}%
  \def\@tempb{}%
  \ifx\@tempa\@tempb
	\else \\\underline{Alternate Supervisor:}\vspace{0.35in}\\\dots\dots\dots\dots\dots\dots\dots\\{#1}\\
  \fi
}

\newcommand\isdefinedsig[1]{%
  \edef\@tempa{#1}%
  \def\@tempb{}%
  \ifx\@tempa\@tempb
	\else \\ \dots\dots\dots\dots\dots\dots\dots\\{#1}\\
  \fi
}
\newcommand\isdefinedspinetitle[1]{%
  \edef\@tempa{#1}%
  \def\@tempb{}%
  \ifx\@tempa\@tempb
	\else (Spine title: #1)\\
  \fi
}
\newcommand\coauthor[1]{%
  \edef\@tempa{#1}%
  \def\@tempb{}%
  \ifx\@tempa\@tempb
	\else \newpage \Large Co-Authorship Statement\normalsize\\\indent\\#1\\
  \fi
}

\newcommand\acknowlege[1]{%
  \edef\@tempa{#1}%
  \def\@tempb{}%
  \ifx\@tempa\@tempb
	\else \newpage \Large Acknowlegements\normalsize\\\indent\\#1\newpage
  \fi
}

%\renewcommand{\appendixtocname}{\Huge \textbf{List of Appendices} \normalsize}
\newcommand{\blank}{\hspace{-2mm}}
\newcommand{\super}{Dr. A. I. McLeod} %supervisor
\newcommand{\superj}{Dr. A. Manning} %joint supervisor, if there is one, leave blank if not (lbin)... only one of the three.
\newcommand{\superc}{} %co-supervisor, if there is one, leave blank if not (lbin)
\newcommand{\supera}{} %alternate supervisor, if there is one, leave blank if not (lbin)
\newcommand{\sco}{Dr. W. J. Braun}  %member of supervisory committee
\newcommand{\sct}{Dr. A. Bing}  %other member of supervisory committee (lbin)
\newcommand{\examo}{Dr. Q. Ring}  %examining committee (up to four, if less leave blank)
\newcommand{\examt}{Dr. W. Fing}
\newcommand{\examth}{Dr. G. Hing}
\newcommand{\examf}{}
\newcommand{\department}{Statistics and Actuarial Science}
\newcommand{\degreename}{Masters of Science}
\newcommand{\firstname}{Tom}
\newcommand{\middlename}{}
\newcommand{\lastname}{Smith}
%\renewcommand{\author}[1]{\ifx\empty#1\else\gdef\@author{#1}\fi}
\newcommand{\authorname}{{\firstname} {\middlename} {\lastname}}
\newcommand{\titl}{Computational Statistics: Time Series and Data Mining}
\newcommand{\spinetitle}{Plib}%only if the above is more than 60 characters
\newcommand{\thesisformat}{Monograph} %or Integrated Article
\newcommand{\gyear}{\number\year}
\newcommand{\makecoauthor}{
%Type information about coauthorship here/
I would like to acknowlege my imaginary friend, Jummi for doing all the work.
}
\newcommand{\makeacknowlege} {
%Type in acknowlegements here
}
\newcommand{\listappendixname}{List of Appendices}
\newlistof{myappendices}{app}{\listappendixname}
\newcommand{\myappendices}[1]{%
\addcontentsline{app}{myappendices}{#1}\par}

\renewcommand{\maketitle}
{\begin{titlepage}
   \setcounter{page}{1}
   %% Set the line spacing to 1 for the title page.
   %\begin{spacing}{1}
   \begin{large}
   \begin{center}
      \mbox{}
      \vfill
      {\MakeUppercase{\titl}}\\
      \isdefinedspinetitle{\spinetitle}
      (Thesis format: \thesisformat)\\
      \vfill
      by \\
      \vfill
      {\firstname} \underline{\lastname}\\
      \vfill
      Graduate Program in {\department}\\
      \vfill
		A thesis submitted in partial fulfillment\\
		of the requirements for the degree of\\
		\degreename\\
		\vfill
		The School of Graduate and Postdoctoral Studies\\
		The University of Western Ontario\\
		London, Ontario, Canada\\
		\vfill
      {\copyright} {\authorname} {\gyear}  \\
      \vspace*{.2in}
   \end{center}
   \end{large}
%   \end{spacing}
   \end{titlepage}

}%\maketitle

\newcommand{\makecert}{
   \setcounter{page}{2}
\vfill
\begin{center}
\large
THE UNIVERSITY OF WESTERN ONTARIO\\
School of Graduate and Postdoctoral Studies\\
\vfill
\textbf{CERTIFICATE OF EXAMINATION}
\end{center}

\vfill
\begin{table}[ht]
\begin{minipage}[t]{0.5\linewidth} %tabular instead?
\begin{tabular}{l}
\underline{Supervisor:}\vspace{0.35in}
\isdefinedsig{\super}
\isco{\superc}
\isjoint{\superj}
\isalt{\supera}
\\
\underline{Supervisory Committee:}\vspace{0.35in}
\isdefinedsig{\sco}\vspace{0.15in}
\isdefinedsig{\sct}
\end{tabular}
\vfill
\end{minipage}
\hspace{0.5in}
\begin{minipage}[t]{0.5\linewidth}
\begin{tabular}{l}
\underline{Examiners:} \\\vspace{.5cm}
\isdefinedsig{\examo}\\
\isdefinedsig{\examt}\\
\isdefinedsig{\examth}\\
\isdefinedsig{\examf}
\end{tabular}
\vfill
\end{minipage}
\vfill
\end{table}
\vfill
\begin{center}
The thesis by \\ \vfill
\textbf{\firstname{} \middlename{} \underline{\lastname}}\\
\vfill
entitled:\\\vfill
\textbf{\titl}\\\vfill
is accepted in partial fulfillment of the \\
requirements for the degree of\\
\degreename\\
\end{center}
\begin{table}[ht]
\begin{minipage}[t]{0.5\linewidth}
\begin{tabular}{l}
\dots\dots\dots\dots\dots\\
Date
\end{tabular}
\end{minipage}
\hspace{0.5in}
\begin{minipage}[t]{0.5\linewidth}
\begin{tabular}{l}
\dots\dots\dots\dots\dots\dots\dots\dots\dots\dots\\
Chair of the Thesis Examination Board
\end{tabular}
\end{minipage}
\end{table}

}

\makeatother
\begin{document}


% Set the page style and numbering for preliminary sections
\begin{preliminary}

% Generate title page from the information in westerntemplate.tex
\maketitle
\addcontentsline{toc}{chapter}{Certificate of Examination}
\makecert
\newpage
%\addcontentsline{toc}{chapter}{Co-Authorship Statement}
%\coauthor{\makecoauthor}  %comment this out if none
%\newpage
\addcontentsline{toc}{chapter}{Acknowlegements}
\acknowlege{\makeacknowlege}	%as above

% Abstract
\addcontentsline{toc}{chapter}{Abstract}
\Large\begin{center}\textbf{Abstract}\end{center}\normalsize
%%  ***  Put your Abstract here.   ***
%% (150 words for M.Sc. and 350 words for Ph.D.)

This is a really silly abstract.

\vfill
\textbf{Keywords:} Time series analysis, data mining


% ToC
\newpage
\tableofcontents\newpage

% List of Figures
\newpage
\addcontentsline{toc}{chapter}{List of Figures}
\listoffigures\newpage

% List of Tables
\addcontentsline{toc}{chapter}{List of Tables}
\listoftables\newpage

% List of Appendices
\addcontentsline{toc}{chapter}{List of Appendices}
\listofmyappendices\newpage

% List of Abbreviations
\addcontentsline{toc}{chapter}{List of Abbreviations}
\printglossary[title = {List of Abbreviations}]
\newpage

%% End of the preliminary sections: reset page style and numbering.
\end{preliminary}

%-------------------------------------------------------------------------------
% Main document
%-------------------------------------------------------------------------------

%%%%%%%%%%%%%%%%%%%%%%%%%%%%%%%%%%%%%%%%%%%%%%%%%%%%%%%%%%%%%%%%%%%%%%%%
%%                                                                    %%
%%                    ***   I M P O R T A N T   ***                   %%
%%                                                                    %%
%% Put your Chapters here; the easiest way to do this is to keep each %%
%% chapter in a separate file and \include all the files right here.  %%
%% Note that each chapter file should start with the line             %%
%% "\chapter{ChapterName}".  Note that using "\include" instead of    %%
%% "\input" makes each chapter start on a new page.                   %%
%%%%%%%%%%%%%%%%%%%%%%%%%%%%%%%%%%%%%%%%%%%%%%%%%%%%%%%%%%%%%%%%%%%%%%%%

\chapter{Introduction and Literature Review}

Antibiotics are chemicals that are used to treat and prevent bacterial infections.
The first antibiotics were isolated in the early 20th century from environmental bacteria and fungi and were adopted quickly into human medicine \parencite{Hutchings.2019}.
At the same time, antibiotics were used for chemotherapy, prophylaxis, and growth promotion in animal agriculture \parencite{Kirchhelle.2018}.
From a year to a couple of decades after each antibiotic reached the drug market, however, resistance was acquired in a bacterium which was historically susceptible \parencite{Ventola.2015}.
To make matters worse, we are in an antibiotic discovery void \parencite{Silver.2011}:
the most recent antibiotic drug class to be discovered, the acid lipopeptides, was reported in 1987, and novel antibiotics that have been reported since then are members of existing drug classes \parencite{Debono.1987}.

Acquired antibiotic resistance is estimated to have caused 5,400 Canadian fatalities in 2018 --- a number which is expected to rise to 13,700 deaths per year by 2050, resulting in a cumulative gross domestic product decline of \$388 billion \parencite{Finlay.2019}.
By 2050, the number of deaths globally due to multidrug-resistant microbial infections is estimated to overtake those caused by road traffic accidents and cancer combined \parencite{ONeill.2016}. Despite, the relatively recent industrialized use of antibiotics in healthcare and agriculture, antibiotic resistance is a modern crisis of ancient origin \parencite{DCosta.2011}.
To ensure the continued efficacy of our existing antibiotics, we must understand the origins of antibiotic resistance and the factors which contribute to increased antibiotic resistance in clinically relevant bacteria.

\section{Antibiotics as a global pollutant}

The mass consumption of antibiotics beginning in the mid-20th century coincides with rising antibiotic resistance in zoonotic pathogens \parencite{Kirchhelle.2018, Ventola.2015} and environmental bacteria \parencite{Madueno.2018}.
In addition to their critical role in human medicine, antibiotics are used for chemotherapy and prophylaxis in farm animals, and were historically fed \textit{en} masse to food-producing animals as growth promotion agents \parencite{Kirchhelle.2018, Witte.1998}.
The use of antibiotics as growth promotion agents has only recently been banned in several countries such as the United States in 2017 \parencite{Scott.2019}, Canada in 2018 \parencite{Finlay.2019}, and China in 2020 \parencite{Hu.2020}, but this practice still continues in many countries with few restrictions on usage \parencite{Chuanchuen.2014}.
The industrialized use of antibiotics in healthcare and agriculture continues to require mass production, which allows antibiotics to enter the environment through many pathways, including discharge from antibiotic manufacturing facilities and hospitals \parencite{Marathe.2019, Bielen.2017}, municipal sewage \parencite{Parnanen.2019}, aquaculture \parencite{Reverter.2020}, and animal agriculture \parencite{Kirchhelle.2018}.
Antibiotic pollution in the environment selects for antibiotic resistance genes \parencite{Lau.2020, Jechalke.2014, Bielen.2017, Yi.2019} which could be transferred to the human microbiome through the interconnected health of humans, animals, and the environment \parencite{Berendonk.2015, HernandoAmado.2019, Tiedje.2019, Robinson.2016}.

\section{Antibiotic resistance: A modern crisis of ancient origin}

Antibiotic resistance is ancient and ubiquitous in the environment \parencite{DCosta.2011, Dunivin.2019}.
For as long as bacteria and fungi have produced antibiotics, antibiotic resistance mechanisms were necessary as a defence against these toxins \parencite{Cundliffe.1989}.
Soil is one of the largest known reservoirs of environmental antibiotic resistance \parencite{Dunivin.2019}.
Soil bacteria are in a state of perpetual chemical warfare and use antibiotics to compete for valuable nutrients such as carbon and nitrogen but may also use them for cellular signalling \parencite{Traxler.2015, Fajardo.2008}.
The saturation of antibiotics in soil has led to an impressive arsenal of antibiotic resistance genes which are currently known to encode resistance to over a dozen antibiotic drug classes \parencite{Wright.2007, Dunivin.2019}.
Antibiotic resistance genes have been sequenced from 30,000 year-old permafrost, and some extant resistance gene families, such as serine beta-lactamases, have been predicted to share the same function as their ancestral sequences from two billion years ago \parencite{DCosta.2011, Hall.2004}.
Because of this conservation of function and continued selection due to antibiotic production, the totality of antibiotic resistance genes in soil --- the soil “resistome” --- is incredibly diverse, and can be selected for by anthropogenic antibiotic pollution \parencite{Lau.2020, Jechalke.2014}.

\section{One Health as a way forward}

“One Health” is a framework that describes the interconnectedness of human, animal, and environmental health, and has been adopted by global health organizations, nations, and researchers to help understand and mitigate the crisis of acquired antibiotic resistance \parencite{Tiedje.2019}.
In 2015, the World Health Organization released their Global Action Plan on Antimicrobial Resistance which identified an important knowledge gap of “understanding how resistance develops and spreads, including how resistance circulates within and between humans and animals and through food, water and the environment” \parencite{WorldHealthOrganization.2015}.
In this Action Plan, the World Health Organization recommended that individual member nations establish national action plans on antimicrobial resistance by adopting the One Health approach to mitigate resistance.
Canada’s Federal Framework for Action established the Canadian Antimicrobial Resistance Surveillance System (CARSS) to expand antimicrobial resistance surveillance to a national level, and in the CARSS 2020 report, the federal government acknowledged that “there is limited data regarding environmental surveillance — a necessary component of any One Health framework” \parencite{PublicHealthAgencyofCanada.2014, PublicHealthAgencyofCanada.2020}.
Of the three pillars of the One Health framework, the role of the environment in clinically relevant antibiotic resistance continues to be the least understood \parencite{Robinson.2016}.

\subsection{The shared human-soil resistome}

Under the One Health framework, anthropogenically-driven increases of antibiotic resistance in soil bacteria may pose a threat to human health due to the shared human-soil resistome \parencite{Forsberg.2012}.
The human microbiome and human bacterial pathogens share antibiotic resistance genes with environmental bacteria \parencite{Forsberg.2012, Smillie.2011, Pal.2016}, but the frequency and context of this exchange is poorly understood \parencite{Berendonk.2015, Huijbers.2015} due to the challenges associated with source attribution --- i.e. determining the exact pathway of a resistance gene from environment to the human microbiome \parencite{Tiedje.2019, Li.2018}.
In a bioinformatics analysis involving 200 soil and 100 human gut metagenomes, 25\% of gut-associated antibiotic resistance genes (\textit{n} = 12) were shared with resistance genes found in soil \parencite{Pal.2016}.
This in silico work is also supported by functional metagenomics studies which have recovered resistance genes in soil bacteria that are identical or very similar to those detected in clinical isolates \parencite{Forsberg.2012, Lau.2017b, Allen.2009}.

Antibiotic resistance genes may be transmitted from soil bacteria to the human microbiome through the consumption of produce \parencite{Maeusli.2020, Blau.2018}.
The transmission of antibiotic resistance from an environmental bacterium on a leafy vegetable, to \textit{Escherichia coli}, and then to a commensal gut bacterium has recently been demonstrated in the mouse microbiome which is a useful model for the human microbiome \parencite{Maeusli.2020, Krych.2013}.
Multi-drug resistance plasmids containing tetracycline, beta-lactam, sulfonamide, aminoglycoside, and fluoroquinolone resistance genes have also been captured in \textit{E. coli} from bacteria in cilantro and mixed salad, indicating that this process could occur in the human gut \parencite{Blau.2018}.
In addition, vegetables grown in soil enriched with antibiotic resistant bacteria can themselves be enriched with the same antibiotic resistance genes \parencite{Murray.2019, Rahube.2016, Rahube.2014}.
Overall, transmission of antibiotic resistance genes from soil bacteria to the human microbiome is plausible, but more research is needed to determine the frequency and mechanisms of this transmission.

\section{The interaction of the soil bacterial resistome and mobilome}

The soil bacterial resistome is generally considered to be structured by bacterial community composition as most antibiotic resistance genes in soil bacteria are embedded within the bacterial chromosome and are therefore inherited vertically \parencite{Dunivin.2019, Forsberg.2014}.
A high soil bacterial diversity has been proposed to “act as a biological barrier” for increased antibiotic resistance as a loss in soil bacterial species diversity is correlated with increased antibiotic resistance gene abundance \parencite{vanGoethem.2018, Chen.2019c, Vivant.2013}.
When a selective pressure (e.g. antibiotics) is strong enough, however, the soil resistome could become ‘decoupled’ from bacterial community composition and diversity as antibiotic resistance genes can be exchanged and re-arranged horizontally \parencite{Johnson.2016}.

\subsection{Mobile genetic elements and horizontal gene transfer}

Mobile genetic elements are entities that promote the mobility of DNA sequences within (chromosome–plasmid, plasmid–plasmid, chromosome–chromosome) and between bacterial genomes, and the totality of all mobile genetic elements in an environment is referred to as the “mobilome” \parencite{Partridge.2018, Perry.2013}.
The mobilome facilitates the horizontal gene transfer of antibiotic resistance genes between bacteria, and includes elements such as plasmids and transposons \parencite{Partridge.2018}, bacteriophages \parencite{Subirats.2016}; \parencite{ColomerLluch.2011}, and membrane vesicles \parencite{Chattopadhyay.2015}.
Horizontal gene transfer occurs through three main mechanisms: conjugation (physical interaction between bacteria), transformation (intake of extracellular DNA), and transduction (phage-mediated) \parencite{Partridge.2018}.
While all three of these mechanisms are known to occur in soil, conjugation has been studied the most extensively and is the most frequent mechanism of horizontal transfer of antibiotic resistance in soil \parencite{Perry.2013}, though transformation and transduction likely also play important roles \parencite{Perry.2013, Aminov.2011}.
Of all of the known non-plasmid mobile genetic elements to mobilize antibiotic resistance in soil, integrons may be the most genetically diverse \parencite{Ghaly.2019}.

\subsection{Integrons}

Integrons are mobile genetic elements that are capable of acquiring, expressing, and re-arranging antibiotic resistance genes within their environments, but notably lack the capability to move their selves, relying upon other mobile genetic elements such as plasmids and transposons for mobility \parencite{Gillings.2014}.
Integrons sample their environment for gene cassettes \parencite{Ghaly.2020} --- pieces of DNA that usually contain one open reading frame followed by a cassette-associated recombination site (\textit{attC}).
Gene cassettes carry a diverse repertoire of antibiotic resistance genes, putative virulence genes, and many other genes of unknown function that have been proposed as a discovery platform for potentially novel natural products \parencite{Ma.2017, Ghaly.2019, Ghaly.2020}.
In a recent sequencing study of the gene cassette metagenomes of soil samples from Antarctica and Australia, it was estimated that there are 4,000 to 18,000 unique gene cassettes per 0.3 g of soil \parencite{Ghaly.2019}.
\todo[color=orange!30]{Vera:
			The leading researchers in the field refer to the totality of gene cassettes in the environment as the "cassette metagenome", so I think it'd be a good idea to keep this terminology for consistency with the literature.
			E.g. \url{http://www.ncbi.nlm.nih.gov/pubmed/31948729}
}

Integrons are characterized by i) an integron-integrase gene (intI) encoding a site-specific tyrosine recombinase, ii) an integron-associated recombination site (\textit{attI}), where incoming gene cassettes are inserted with the help of IntI, and iii) an integron-associated promoter (P\textsubscript{C}) which expresses downstream gene cassettes (Figure \ref{fig:class-1-integron-structure}) \parencite{Gillings.2014}.
IntI catalyzes the recombination of \textit{attC} with \textit{attI} to insert an incoming gene cassette downstream of P\textsubscript{C} and can also reversibly excise an integrated gene cassette from the integron structure.
The recombination event produces two daughter molecules: a duplicate of the original integron structure, and the other with the integrated gene cassette \parencite{Ghaly.2020}.
This phenomenon allows the host bacterium to sample each gene cassette for fitness tradeoffs prior to stable integration, and in the context of a cassette-embedded antibiotic resistance gene, maintain the antibiotic resistance phenotype if it confers a selective advantage \parencite{Ghaly.2020}.

\begin{figure}[htpb]
	\centering
		\includegraphics[width=0.75\textwidth]{figures/integron/class-1-integron-structure_v03.pdf}
	\caption[Structure of a class 1 integron.]{
		Class 1 integrons are clinically-relevant integrons that are uniquitous in human-impacted environments, and are characterized by their class 1 integron-integrase gene (\textit{intI1}) and 3' conserved sequence (3'-CS) which contains a partially deleted biocide resistance gene (\textit{qacE$\Delta$1}) and a sulfonamide antibiotic resistance gene (\textit{sul1}).
		Cassette-associated recombination is similar for all classes of integrons:
		\textbf{1}) \textit{intI1} is expressed by the integron-integrase promoter P\textsubscript{\textit{intI1}},
		\textbf{2}) the tyrosine recombinase IntI1 catalyzes the recombination of the incoming gene cassette's attachment site (\textit{attC}) with the integron-associated attachment site (\textit{attI}),
		\textbf{3}) the integron-associated promoter (P\textsubscript{C}) expresses the inserted gene cassette, and
		\textbf{4}) another gene cassette may be inserted to form a gene cassette array of the two cassettes.
		This process may continue for up to eight gene cassettes.
	}
	\label{fig:class-1-integron-structure}
\end{figure}

\subsection{Co-selection}

Mobile genetic elements, especially integrons, facilitate the co-selection of antibiotic resistance genes in soil \parencite{Pal.2015}.
Co-selection occurs when antibiotic exposure results in increased resistance to an environmentally absent antibiotic drug class.
Co-selection can be explained through two main processes: i) cross-resistance, when an antibiotic resistance gene is selected by an environmentally present drug class and also confers resistance to an absent drug class; and ii) co-resistance, when an antibiotic resistance gene is selected and is physically linked to a different resistance gene which confers resistance to an absent drug class \parencite{Wales.2015}.
Class 1 integrons, which are known to possess gene cassettes that are heavily biased towards conferring antibiotic resistance phenotypes \parencite{Ghaly.2020, Yang.2021}, facilitate the co-selection of antibiotic resistance genes in the environment by forming multi-drug resistance gene cassette arrays \parencite{Naas.2001}.
Furthermore, class 1 integrons form linkage clusters of antibiotic resistance in soil, as they frequently co-occur with other mobile genetic elements and with antibiotic resistance genes that are not embedded within gene cassettes \parencite{Johnson.2016, Pal.2015}.

\subsection{Class 1 integrons}

Of the hundreds of different classes of integrons \parencite{Abella.2015}, the class 1 integron is the most prolific in human pathogens and is also abundant in soil \parencite{Dawes.2010, RuizMartinez.2011, Gillings.2018}.
Class 1 integrons typically carry less than six and no more than eight gene cassettes \parencite{Gillings.2014, Naas.2001}.
Class 1 integrons are distinguished from other classes of integrons by their \textit{intI1} gene known as the class 1 integron-integrase, which are 98\% identical in amino acid sequence \parencite{Roy.2021}.
The “clinical” or “\textit{sul1}-type” variant of class 1 integrons has a 3’ conserved segment with a partially deleted but semi-functional disinfectant resistance gene \textit{qacE${\Delta}$1}, followed by the sulfonamide antibiotic resistance gene \textit{sul1} (Figure \ref{fig:class-1-integron-structure}) \parencite{Partridge.2018}.
From a One Health perspective, class 1 integrons are of particular concern because
i) they have become endemic to human and environmental microbiomes \parencite{Gillings.2017},
ii) they are increased in the presence of antibiotic pollution \parencite{Gillings.2017, Wright.2008, Stalder.2014},
iii) their gene cassette content is biased towards conferring antibiotic resistance phenotypes \parencite{Yang.2021},
iv) some antibiotics indirectly increase the transcriptional activity of \textit{intI1}, thereby promoting gene cassette recombination \parencite{Baharoglu.2010}, and
v) they form co-occurrence linkage clusters with other mobile genetic elements and antibiotic resistance genes \parencite{Pal.2015}.
Class 1 integrons are known to be enriched in soils that have been polluted with macrolide antibiotics \parencite{Lau.2020}.

\section{Macrolide antibiotics}

\subsection{Importance to human and animal medicine}

Macrolide antibiotics are the third most-consumed antibiotics in Canada and are used as first-line treatments for serious diseases such as community acquired pneumonia (\textit{Streptococcus} and \textit{Mycoplasma}), campylobacteriosis (\textit{Campylobacter jejuni}, \textit{C. coli}), and as alternatives for individuals allergic to beta-lactams \parencite{PublicHealthAgencyofCanada.2020, CapeloMartinez.2019}.
Despite their prolific use in human medicine, most macrolide antibiotics that are sold are consumed by food-producing animals for chemotherapy and prophylaxis \parencite{CapeloMartinez.2019}.
In 2018 alone, 87,221 kg of macrolide antibiotics were sold for consumption in Canadian agriculture \parencite{PublicHealthAgencyofCanada.2020}.
These antibiotics have been deemed “critically important” for human medicine by the World Health Organization and resistance to these drugs is rising \parencite{Resistance.2017, PublicHealthAgencyofCanada.2020}.
Risk management strategies that focus on reducing macrolide presence in the environment will mitigate future risks to human health.

\subsection{Structure}

\begin{figure}[htb]
	\centering
		\includegraphics[width=0.7\textwidth]{figures/macrolide-chemical-structures/macrolide-chemical-structures.png}
	\caption{Chemical structures of erythromycin, clarithromycin, and azithromycin.}
	\label{fig:macrolide-chemical-structures}
\end{figure}

Erythromycin A was first isolated from the soil bacterium \textit{Saccharopolyspora erythraea} in 1952 and most other macrolide antibiotics are chemically modified derivatives of erythromycin A, which is the primary active compound in the antibiotic medicine erythromycin \parencite{Haight.1952}.
Erythromycin, clarithromycin, and azithromycin are the most consumed macrolides in human medicine, as reflected by their prevalence in wastewater \parencite{Miao.2004, RodriguezMozaz.2020}.
Macrolide antibiotics are characterized by a 14-, 15-, or 16-membered macrocyclic lactone ring bound to at least one deoxy sugar (erythromycin A, clarithromycin, and azithromycin are bound to desosamine and cladinose) (Figure \ref{fig:macrolide-chemical-structures}) \parencite{CapeloMartinez.2019}.
Clarithromycin is identical to the 14-membered erythromycin A but with a methylated C6-hydroxy group, resulting in a more acid-labile molecule.
Azithromycin is a 15-membered macrolide created from the insertion of a nitrogen atom into the lactone ring of erythromycin A, resulting in more potent antibacterial activity against many gram-negative pathogens such as \textit{Haemophilus influenzae} (bacterial flu) and \textit{Neisseria gonorrhoeae} (gonorrhea) \parencite{Yanagihara.2009}.

\subsection{Mechanisms of action and resistance}

Macrolides inhibit protein synthesis in gram-positive (and some gram-negative) bacteria by reversibly binding to the 23S ribosomal RNA (rRNA) within the bacterial 50S ribosomal subunit, at the entrance of the peptide exit tunnel, which imperfectly prevents assembly and elongation of the peptide \parencite{CapeloMartinez.2019, Fyfe.2016}.
This mechanism is usually bacteriostatic --- the macrolides alone do not kill all of the bacterial cells and the host’s immune system must clear the remainder of the infection \parencite{Pankey.2004}.
Macrolide antibiotic resistance mechanisms in bacteria are diverse \parencite{Fyfe.2016}.
Resistance can be evolved through target site mutation in the ribosome or can be horizontally acquired:
Antibiotic resistance genes may encode a methyltransferase which methylates the ribosome and prevents binding of the antibiotic (erm gene family), or an efflux pump to remove the antibiotic from the cell (msr and mef gene families), or a phosphotransferase to inactivate the antibiotic (mph gene family) \parencite{Fyfe.2016}.
Many of these antibiotic resistance genes are mobile as demonstrated by the erm gene family, as over 40 erm genes have been identified and most of them are plasmid-encoded \parencite{Alcock.2020, Leclercq.2002}.

\subsection{Effects of long-term macrolide antibiotic pollution in agricultural soil}

Macrolide antibiotic pollution of soil is known to promote antibiotic resistance \parencite{Lau.2020}.
Over an eight-year period, soil field plots were annually exposed to the macrolide antibiotics erythromycin, clarithromycin, and azithromycin which resulted in increased abundances of antibiotic resistance genes and mobile genetic elements, including class 1 integrons.
Interestingly, most of the antibiotic resistance genes that were increased were predicted to confer resistance to non-macrolide antibiotic drug classes, indicating that macrolide antibiotic exposure of soil co-selects for resistance to aminoglycosides, sulfonamides, and trimethoprim.
Several of these antibiotic resistance genes are known to be associated with class 1 integrons, suggesting a role for class 1 integrons in this co-selection process \parencite{Lau.2020}.
Macrolide antibiotics are also more rapidly degraded in soil with a previous exposure history to macrolides, indicating that macrolides may have an effect on soil microbial diversity and composition \parencite{Topp.2016}.
This effect could be ecotoxic in nature and could represent a threat to agricultural productivity \parencite{Prashar.2014}.

\section{Biosolids as a vector for macrolide antibiotic pollution of soil}

Macrolide antibiotics are discharged into the environment through human waste and are inefficiently removed by most wastewater treatment processes \parencite{LeMinh.2010, Luo.2014}.
In Canada, only 28\% of the population is served by tertiary wastewater treatment which removes greater quantities of macrolides than other treatments, and the focus of this treatment is on disinfection rather than the removal of pharmaceuticals \parencite{EnvironmentandClimateChangeCanada.2020, LeMinh.2010}.
Abundances of antibiotics and antibiotic resistance genes are currently unregulated in Canadian wastewater effluent and many other countries, and as a result, wastewater effluent is also a hotspot of antibiotic resistance genes and mobile genetic elements \parencite{Rizzo.2013, Che.2019}.
Macrolide antibiotics from wastewater effluent can contaminate soil through the agricultural use of treated sewage sludge \parencite{McClellan.2010, Sabourin.2012}.

\subsection{Agricultural use of biosolids}

Biosolids (treated sewage sludge) are recycled material from wastewater treatment plants that can be used as an agricultural fertilizer and soil amendment \parencite{Sharma.2017}; the solid portion of biosolids is comprised of approximately 50\% organic matter and 50\% mineral material \parencite{OntarioMinistryofAgricultureFoodandRuralAffairs.2010}.
Unfortunately, antibiotics that survive the wastewater treatment process can carry over into biosolids, including those of the macrolide antibiotics drug class such as erythromycin, clarithromycin, and azithromycin \parencite{McClellan.2010, Sabourin.2012, Chenxi.2008}.
Biosolids are produced from the separation of wastewater into water and solids, followed by treatment of the solid portion to reduce pathogens and odour using a combination of chemical, biological, or physical processes \parencite{LeMinh.2010}.
Biosolids improve soil quality and fertility:
soil that is more fertile requires less inorganic fertilizer, which reduces the risk of fertilizer runoff into adjacent water sources, and soil that has more organic matter has increased moisture retention.
Biosolids are applied to agricultural soil on every continent except Antarctica, but usage is highly variable:
almost all of the biosolids that are produced in the United Kingdom (78\%) and Ireland (96\%) are land-applied, whereas only 55\% are land-applied in the United States \parencite{Sharma.2017}.
There are concerns, however, that the long-term application of biosolids to agricultural soil could introduce macrolide antibiotics into the environment and promote resistance in soil bacteria, which could be transferred to humans via consumption of produce under the One Health framework \parencite{Lau.2020, Sabourin.2012}.

\subsection{Concentrations of macrolide antibiotics in biosolids and comparison to PNEC}

In a survey of 74 locations producing treated biosolids in the United States, the 95\textsuperscript{th} percentile concentrations of detected macrolides were 0.1 mg kg\textsuperscript{-1} biosolids (dry weight) erythromycin, 0.2 mg kg\textsuperscript{-1} clarithromycin, and 3.2 mg kg\textsuperscript{-1} azithromycin \parencite{U.S.EnvironmentalProtectionAgency.2021}.
Biosolids are typically applied at a rate of 1\% dw dw\textsuperscript{-1} soil, meaning that the upper-range of environmentally relevant concentrations for these macrolides in biosolids-applied soil would be 0.001 mg kg\textsuperscript{-1} erythromycin, 0.002 mg kg\textsuperscript{-1} clarithromycin, and 0.032 mg kg\textsuperscript{-1} azithromycin \parencite{Sidhu.2021}.
These concentrations are equal to, 8-fold, and 128-fold greater than the Predicted No-Effect Concentrations (PNEC) for erythromycin, clarithromycin, and azithromycin in surface water, as determined by \cite{BengtssonPalme.2016}.
The PNEC is the concentration above which antibiotic resistance could be selected for in environmental bacteria and have been proposed as limits for the regulation of antibiotics in the environment.
The 95\textsuperscript{th} percentile concentrations of macrolides in municipal biosolids are equal to or exceed those for \textcolor{red}{freshwater}, and biosolids could therefore realistically select for antibiotic resistance in land-applied soil.
\todo[color=blue!20]{I think most PNECs are based on aqueous environments, no experimentally deduced soil PNECs...}

% \todo[color=red!20]{Ed:
% I used Johan's paper for estimation of the PNECs, but I believe these estimates were for surface water.
% I found an alternative resource for PNECs which apply an equation to the surface water value to estimate the soil PNEC (NORMAN Ecotoxicology Database, \url{https://www.norman-network.com/nds/ecotox/}). This database was used by a paper that Thomas Berendonk was on (\url{https://doi.org/10.1016/j.envint.2020.105597}).
% Do you think I should use this database over Johan's values?
% }

\subsection{Critical knowledge gaps}

The land-application of biosolids introduces antibiotics into agricultural soil that have carried over from the wastewater treatment process \parencite{McClellan.2010, Sabourin.2012}.
These antibiotics are present at concentrations that are predicted to select for resistance in the soil bacterial community \parencite{U.S.EnvironmentalProtectionAgency.2021, BengtssonPalme.2016}, and the exposure of soil to macrolide antibiotics increases the abundance of antibiotic resistance genes and mobile genetic elements in soil bacteria, including class 1 integrons \parencite{Lau.2020}.
This antibiotic exposure is also known to co-select for resistance to anthropogenically absent drug classes of antibiotics and resistance genes that are known to be associated with class 1 integron gene cassettes \parencite{Lau.2020}.

The effects of macrolide antibiotic exposure on the soil bacterial community and the integron gene cassette metagenome remain to be determined, as does the potential for macrolides to select for antibiotic resistance in soil bacteria at concentrations that are environmentally relevant to a biosolids exposure scenario.
Because the health of humans and soil are interconnected under the One Health framework \parencite{Tiedje.2019}, and because biosolids are a vector for the introduction of macrolide antibiotics into the environment \parencite{Sabourin.2012, McClellan.2010}, we must determine the consequences of long-term macrolide exposure on the development of antibiotic resistance in soil bacteria in order to assess if the repeated use of biosolids in agriculture may pose a risk to human health.

\section{Review of sequencing-based methods}

\subsection{16S rDNA sequencing}

Most soil bacteria are uncultivable:
of the approximately 10\textsuperscript{8} cells of bacteria that can be found in a single gram of bulk soil, less than 1\% are estimated to be cultivable using standard growth techniques \parencite{Raynaud.2014, vanPham.2012}.
The selectivity of nutrient media, competition in media by faster growing organisms, and low abundance in the environment relative to other species all contribute to the difficulty in culturing most soil bacteria \parencite[229–300]{vanElsas.2019b}.
Sequencing-based approaches to investigate the bacterial community have shone a light on the incredible diversity of uncultivable soil bacteria \parencite{Hug.2016} and have allowed researchers to investigate the responses of the soil bacterial community to environmental perturbations \parencite{Isobe.2019, Isobe.2020}.
Of the different sequencing-based approaches available to investigate soil bacterial community composition, 16S rDNA sequencing and metagenomic sequencing are presently the most common.

16S rDNA sequencing involves the targeted amplicon sequencing of the 16S rRNA gene.
The bacterial 16S rRNA gene is used to determine bacterial taxonomy due to i) regions of highly conserved sequence between bacterial species and ii) hypervariable regions which allow for species-specific classification \parencite{vanPham.2012}.
Typically, only a subset of the hypervariable regions are sequenced (usually some combination of the V3, V4, V5, V6 regions) to classify bacterial taxa \parencite{Yang.2016}, though advances in long-read technologies have made full-length 16S rDNA sequencing an attractive alternative \parencite{Shin.2016, Numberger.2019}.
First, total genomic DNA is isolated from the soil sample and the hypervariable regions (the 16S rDNA) of the bacterial 16S rRNA gene are PCR amplified using site-specific primers.
Next, a DNA library is prepared from the resulting amplicons and the library is subsequently sequenced.
Finally, the biological sequence data that are generated from the sequencer are analyzed using bioinformatics software.
Taxonomic classification software such as QIIME 2 can cluster sequence reads based on dissimilarity thresholds into operational taxonomic units, or software such as DADA2 can attempt to infer biological sequences prior to PCR and sequencing to construct amplicon sequence variants \parencite{Bolyen.2019, Callahan.2016}.
The use of amplicon sequence variants over operational taxonomic units is preferred, as sequence variants attempt to deal with sequencing errors and better reflect the DNA that was actually sequenced \parencite{Callahan.2017}.

\subsection{Metagenomic sequencing}

Metagenomic sequencing, the non-selective sequencing of the total genomic DNA in an environment, is another popular approach for determining bacterial community composition in soil.
In metagenomic sequencing, total genomic DNA is isolated from the soil, a DNA library is prepared from the total genomic DNA, and the DNA library is then sequenced.
Metagenomic sequencing has several advantages over 16S rDNA sequencing for determining bacterial community composition:
16S rDNA sequencing suffers from primer bias during PCR, as the primers amplify different ribosomal sequences with different efficiencies, resulting in a bias of sequence reads to taxa with rRNA genes that are more similar to the primer-binding site \parencite{Tremblay.2015}.
In addition, metagenomic sequencing generates data covering multiple genes and possibly entire bacterial genomes, allowing for a metagenomic functional analysis in addition to taxonomic analysis \parencite{Li.2015c}.
At present, the greatest downside to metagenomic sequencing is the higher financial cost associated with metagenomic sequencing compared to 16S rDNA sequencing as a greater sequencing depth is required in order to achieve a detailed picture of the bacterial community \parencite{Scholz.2012}.

Metagenomic sequencing can also be used to identify antibiotic resistance genes and mobile genetic elements within a bacterial community \parencite{Boolchandani.2019}.
Metagenomic sequencing confers many advantages over other methods for studying antibiotic resistance in soil bacteria.
Antibiotic resistance genes and mobile genetic elements are distributed among diverse soil bacterial taxa --- many of which are difficult to cultivate under normal laboratory conditions \parencite{Dunivin.2019}.
Metagenomic sequencing, compared to PCR-based methods, also allows for the discovery of novel antibiotic resistance genes and mobile genetic elements for which PCR primers have not been developed or are not available \parencite{Boolchandani.2019}.
Antibiotic resistance genes and mobile genetic elements can be identified in metagenomic sequence data by aligning sequence reads to databases of known antibiotic resistance genes and mobile genetic elements, such as the Comprehensive Antibiotic Resistance Database (CARD) \parencite{Alcock.2020}.
Other bioinformatics software, such as Metagenomic Markov models for Antimicrobial Resistance Characterization (Meta-MARC), use machine learning principles to identify novel antibiotic resistance genes from metagenomic sequence data \parencite{Lakin.2019}.

\subsection{Integron gene cassette sequencing}

Integrons can be identified in metagenomic datasets using software that scans for \textit{intI1} and \textit{attC} sites \parencite{Cury.2016}.
Such software could theoretically be fine-tuned to only target specific classes of integrons or could be made more sensitive to detect novel classes of integrons.
However, the analysis of metagenomic data alone is unlikely to capture the full diversity of integron gene cassettes in a soil sample due to the complexity of the microbiome.
In addition, using sequence alignment software such as BLAST or DIAMOND to search for class 1 integrons by identifying \textit{intI1} wouldn’t capture the diversity of the hundreds of other known classes of integrons \parencite{Altschul.1990, Buchfink.2015}.
The targeted amplicon sequencing of integron gene cassettes is a PCR-based approach that can be used to characterize the diversity of integron gene cassettes in any environment:
PCR primers can be designed to target the \textit{attC} or \textit{attI} sites and/or the integron-integrase gene to amplify gene cassettes within a specific integron class or within diverse environmental integrons, and similar to 16S rDNA amplicons, these amplicons can then be sequenced and analyzed using bioinformatics software for antibiotic resistance gene identification \parencite{Yang.2021, Ghaly.2019}.
Cassette-embedded genes could also be assigned more general functions using databases of orthologous groups such as eggNOG \parencite{HuertaCepas.2019}.

\subsection{Compositional data analysis}

Much statistical software has been developed to help identify biologically meaningful differences in the diversity and compositions of groups from sequence data.
For example, DESeq and edgeR both accept a matrix of samples versus counts as input (also known as a feature table) and then attempt to identify differentially abundant features between groups of samples (e.g. treatments) in the table \parencite{Anders.2010, Robinson.2010}.
This feature table could describe the counts of any genomic feature of interest, including bacterial amplicon sequence variants, antibiotic resistance genes, mobile genetic elements, or cassette-embedded genes.
DESeq and edgeR both assume that sequence reads can be normalized based upon sequence depth (conversion of counts to proportions); however, sequence data is compositional by nature, as sequencing instruments have constrained capacities to sequence samples, and therefore generate counts that can themselves be described as proportions of a constrained, unknown sum \parencite{Gloor.2017}.
More recently, bioinformatics tools such as ALDEx2 and ANCOM with Bias Correction (ANCOM-BC) have been developed which use statistical techniques that are appropriate for identifying differentially abundant features in sequencing datasets \parencite{Fernandes.2014, Lin.2020}.
This software can be used to investigate differences in the compositions of sequence datasets that are relevant to the analysis of soil microbiomes.

\section{Objectives and hypotheses}

Following the observed increased abundances of antibiotic resistance genes and mobile genetic elements in agricultural soil that had been annually exposed to macrolide antibiotics for eight years, the contributions of co-selection and bacterial community composition to these increases remained to be determined, as did the potential for these effects to occur at an environmentally realistic dose for a biosolids land-application scenario \parencite{Lau.2020}.
To further investigate if macrolide antibiotic exposure of soil promotes resistance at an environmentally realistic dose, and to elucidate the mechanisms of increased resistance at an effect-inducing unrealistically high dose, we obtained soil DNA from field plots treated with macrolide antibiotics for ten years and from untreated plots.
The 16S rDNA and class 1 integron gene cassettes were PCR amplified and sequenced, and the total soil metagenome was sequenced.

I hypothesized that long-term macrolide antibiotic exposure of agricultural soil, at both a realistic dose (0.1 mg kg\textsuperscript{-1} soil) and an unrealistically high dose (10 mg kg\textsuperscript{-1}) for biosolids carryover, would affect the composition and diversity of the soil bacterial community, resistome, and mobilome.

I predicted that:

\begin{enumerate}
  \item{Antibiotic resistance genes and mobile genetic elements would increase in response to antibiotic exposure,}
  \item{Bacterial community composition and diversity would differ between antibiotic-exposed and -unexposed soil, and}
  \item{Integron gene cassette composition and diversity would differ between antibiotic-exposed and -unexposed soil.}
\end{enumerate}

\chapter{Methods}

\section{Field experiment}

Soil microplots were established at Agriculture and Agri-Food Canada in London, Ontario as described by \parencite{Topp.2016}.
Briefly, twelve 2 m$^{2}$ fibreglass frames were placed into the ground in 2010 and filled with a silt grey loam soil commonly used in Canadian agriculture.
Each summer for 10 years, these microplots were exposed to a dose of mixed macrolide antibiotics at concentrations of 0.1 mg kg$^{-1}$ soil (low, \textit{n} = 4), 10 mg kg$^{-1}$ (high, \textit{n} = 4), or were left unexposed (\textit{n} = 4).
Stock solutions of erythromycin, clarithromycin, and azithromycin were prepared to 1 mg mL$^{-1}$ in \textcolor{red}{99}\% ethanol and stored at -20\degree C until used.
Each June, antibiotics were mixed into 1 kg of soil obtained from each plot and soil was re-incorporated into the source microplots to a depth of 10 cm using a mechanized rototiller.
Soybean seeds were planted immediately after adding the antibiotics and plots were maintained throughout the growing season by manual weeding only.
In 2019, six 20 cm soil core samples were obtained 30 days post-application, pooled, then sieved to a maximum particle size of 2 mm. Soil was stored at -20\degree C prior to DNA isolation.

\section{DNA isolation, PCR, and library preparation}

Total genomic DNA was isolated from 250 mg of soil from each microplot using the DNeasy PowerSoil Kit (Qiagen) and eluted in 100 \u L of 10 mM Tris-HCl following the manufacturer’s protocol. Spectrophotometric readings of the eluted DNA were taken using a NanoDrop ND1000 microspectrophotometer (NanoDrop Technologies) to assess DNA quality (A260/A280) and a Qubit\textsuperscript{\texttrademark{}} dsDNA HS Assay Kit was used to determine DNA concentration with a Qubit\textsuperscript{\texttrademark{}} 4 Fluorometer (Invitrogen).
DNA was stored at -20\degree C.

\section{Next-generation sequencing}

\subsection{Integron gene cassette sequencing}

Integron gene cassettes were PCR amplified using primers described by \cite{Stokes.2001} with 33 and 34 bp Illumina adapter overhang sequences ligated onto the 5’ ends (Supplementary Table \ref{supp-table:primers}):
The purpose of these 5’ ends was to extend the distance between the tagmentation site and the desired gene cassette sequence, as ~ 50 bp from each distal end of the amplicon was expected to be lost during preparation of the sequencing library due to transposome activity.
The gene cassette PCR primers anneal to highly conserved GTTRRRY motifs within the \textit{attC} recombination sites of gene cassettes \dummyfig.
\todo{A small diagram of where the primers anneal to within an integron (don't use the class 1 integron structure; just use a generic integron).}
Total genomic DNA isolated from the microplot soils were diluted 10-fold in Tris-EDTA buffer and used as template DNA for five technical replicates of 25 \u L PCR reactions (125 \u L total), and amplified under the following thermocycler conditions:
94\degree C for 3 min; 35 cycles of 94\degree C for 30 s, 55\degree C for 1 min, 72\degree C for 2 min 30 s; 72\degree C for 5 min.
Each PCR reaction was comprised of 2 \u L of diluted template DNA, 0.25 \u L of Q5\textsuperscript{\textregistered{}} High-Fidelity DNA Polymerase (New England BioLabs), 0.2 \u L of 25 mM dNTPs, 5 \u L of 5X Q5\textsuperscript{\textregistered{}} Reaction Buffer, and 1.13 \u L of each 10 \u M forward and reverse primer.
Technical replicates were pooled together and PCR product was purified using the GenepHlow PCR Cleanup Kit (Geneaid), and eluted in 25 \u L of nuclease-free water.
DNA concentration of the cleaned PCR product was determined using the Qubit\textsuperscript{\texttrademark{}} dsDNA HS Assay Kit (Invitrogen).

The Nextera\textsuperscript{\textregistered{}} XT DNA Library Preparation Kit was used to prepare DNA libraries of the gene cassette amplicons for sequencing by following the manufacturer’s protocol.
The DNA libraries were indexed using the Nextera\textsuperscript{\textregistered{}} XT Index Kit (Illumina) following the tagmentation step.
Bead purification of the DNA libraries was performed using a 1.8X bead-supernatant ratio of HighPrep\textsuperscript{\texttrademark{}} PCR solution (MAGBIO Genomics), quantified using Qubit\textsuperscript{\texttrademark{}} dsDNA HS Assay Kit (Invitrogen), and sized using the Agilent High Sensitivity DNA Kit on a Bioanalyzer 2100 (Agilent).
Individual libraries were diluted to 10 nM in nuclease-free water and 15 \u L of each diluted library were pooled together for multiplex sequencing.
The pooled DNA library was sent to The Hospital for Sick Children in Toronto, Ontario for 2 x 125 bp sequencing on a HiSeq 2500 instrument (Illumina).

\subsection{16S rDNA amplicon sequencing}

For 16S rDNA sequencing, the total genomic DNA was diluted 10-fold in Tris-EDTA buffer and used as template for PCR amplification of the V3 and V4 regions of the bacterial 16S rDNA gene (Supplementary Table \ref{supp-table:primers}).
The MiSeq Reagent Kit v3 (600 cycle; Illumina) was used to prepare the amplicon libraries, and the libraries were indexed using the Nextera\textsuperscript{\textregistered{}} XT Index Kit (Illumina) by following the manufacturer’s protocol.
The amplicon library was sent to the Canadian Food Inspection Agency (CFIA) in Ottawa, Ontario for 2 x 300 bp sequencing on a MiSeq instrument (Illumina).

\subsection{Metagenomic sequencing}

Only three of four biological replicates were used for downstream metagenomic sequencing.
For metagenomic sequencing, the DNA concentrations of each sample were determined using the Qubit\textsuperscript{\texttrademark{}} dsDNA HS Assay Kit (Invitrogen) and then sent to The Hospital for Sick Children in Toronto, Ontario for library preparation and shotgun sequencing across two lanes on a HiSeq 2500 instrument (Illumina).

\section{Sequence data analysis}

For all sequence datasets, the quality of the demultiplexed reads was assessed using FastQC (v0.11.8) and MultiQC (v1.7) and then re-assessed after adapter removal and quality-based trimming (when applicable) \parencite{Andrews.2010, Ewels.2016}.

\subsection{16S rDNA sequence analysis}

To remove low-quality bases from the 16S rDNA amplicon reads, Trimmomatic (v0.36) was run in paired-end mode:
The first 25 bases of each read were dropped; sliding window trimming was performed where a window of 4 bp would be trimmed if the average quality of the window had a quality score (Q-score) $<$ 15; remaining reads with a length $<$ 25 bp were discarded \parencite{Bolger.2014}.
To denoise the trimmed 16S rDNA amplicon reads, remove chimeric sequences, merge reads, and then establish a set of unique amplicon sequence variants and obtain their counts, the DADA2 denoise-paired plugin  within QIIME 2 (v2019.10) was run with default options and without further truncation of the 5’ and 3’ ends \parencite{Callahan.2016, Bolyen.2019}.
To assign taxonomy to the amplicon sequence variants, the QIIME 2 feature-classifier plugin was first trained with SILVA (v132) 16S rRNA reference sequences using Naïve Bayes classification and was then used to classify the sequence variants \parencite{Quast.2013, Bokulich.2018}.

\subsection{Metagenomic sequence analysis}

Cutadapt (v2.8) was used to remove adapter sequences from the 3’ ends of metagenomic sequence reads \parencite{Martin.2011}.
Trimmomatic was run in paired-end mode to remove low-quality leading and trailing bases from the adapter-trimmed reads (Q-score $<$ 20), and remaining reads with a length $<$ 100 bp were discarded.
MetaPhlAn3 (v3.0.7) was used to assign taxonomy to metagenomic reads using the 'very-sensitive' algorithm of Bowtie2, ignoring eukaryotes and archaea, and profiling the metagenomes as relative abundances with estimation of the number of reads coming from each bacterial clade \parencite{Beghini.2020}.

To identify metagenomic reads that corresponded to antibiotic resistance genes, the metagenomic reads were mapped to two CARD databases:
The CARD ‘canonical’ database (v3.0.8) of phenotypically confirmed antibiotic resistance genes; and the CARD Prevalence, Resistomes, \& Variants (v3.0.7) database of in silico predicted resistance genes, derived from genomic data of 82 human pathogens.
The metagenomic reads were mapped to these databases using Bowtie2 implemented by the CARD Resistance Gene Identifier in metagenomics mode (v5.1.0) \parencite{Alcock.2020}.
To identify metagenomics reads that corresponded to mobile genetic elements, the metagenomic reads were mapped to a database of known mobile genetic elements created by \cite{Parnanen.2018} (downloaded from \url{https://github.com/KatariinaParnanen/MobileGeneticElementDatabase} on 2021-05-24).
The metagenomic reads were mapped to this database using Bowtie2 (v2.4.2) with end-to-end searching and using the pre-defined 'very-sensitive' search algorithm, except for allowing a maximum of one mismatch in the seed alignment \parencite{Langmead.2012}.
Fold-coverages of antibiotic resistance genes and mobile genetic elements were used as abundances for downstream analysis.

\subsection{Integron gene cassette sequence analysis} \label{section:cassette-sequence-analysis}

No quality-based trimming was performed for the integron gene cassette reads to preserve primer-binding sites for downstream filtering, but Cutadapt was used to remove adapter sequences from the 3’ ends of gene cassette reads.
Integron gene cassette sequence reads were assembled into contigs using MEGAHIT (v1.2.9) with default options \parencite{Li.2015c}.
Assembly quality was assessed using BBTools’ Stats (v38.90) \parencite{Bushnell.2016}.
Individual sample assemblies were combined into a master assembly for downstream filtering of gene cassettes.

The highly conserved motifs within integron gene cassette attC sites were used to identify the boundaries of gene cassettes from assembled contigs; if an assembled contig did not contain the terminal 9 bp of both of these motifs, it was identified using BBTools’ BBDuk (v38.90) and discarded from further analysis \parencite{Bushnell.2016}.
Prokka (v1.14.6) was used to identify open reading frames --- putative genes that may encode a protein --- within the surviving gene cassette contigs, which were then clustered at 97\% identity using CD-HIT (v4.8.1) to obtain unique open reading frames \parencite{Seemann.2014, Fu.2012}.

To identify integron gene cassette open reading frames that could correspond to antibiotic resistance genes, the open reading frames were aligned against CARD’s ‘canonical’ protein homolog database using an implementation of BLAST within CARD-RGI, including all ‘loose’ hits and running in low-quality mode.
To further potentiate the discovery of novel antibiotic resistance genes, the translated protein sequences were also scanned using Meta-MARC and hmmer (v3.1b2) with Group I models only (downloaded from \url{https://github.com/lakinsm/meta-marc} on 2021-02-13) \parencite{Lakin.2019, Wheeler.2013}.

We considered the positive identification of integron gene cassette open reading frames as antibiotic resistance genes at three different levels of confidence (high, moderate, low) which were determined by visual inspection of alignment statistics distributions \dummysupfig:
\todo[noinline]{Violin and scatter plots of alignment statistics.}

\begin{itemize}
	\item{High confidence: CARD-RGI ‘strict’ hits; Meta-MARC hits with E-value $\leq$ 1E-10.}
	\item{Moderate confidence: All high confidence hits; CARD-RGI ‘loose’ hits with percent identity $>$ 60 and percent length of reference sequence $>$ 60; Meta-MARC hits with E-value $\leq$ 1E-1.}
	\item{Low confidence: All high and moderate confidence hits; CARD-RGI ‘loose’ hits with percent identity $>$ 40 and percent length of reference sequence $>$ 40; Meta-MARC hits with E-value $\leq$ 1.}
\end{itemize}

To predict general functions for integron gene cassette open reading frames, the open reading frames were scanned for similarity to orthologous groups in the eggNOG database (v5.0) and assigned a Cluster of Orthologous Groups (COG) functional category using the web implementation of eggNOG-mapper (v2.0) \parencite{HuertaCepas.2019}.

To obtain fold-coverages for integron gene cassette open reading frames, BBTools’ BBMap (v38.90) was used to map gene cassette sequence reads back onto unique open reading frames \parencite{Bushnell.2016}.
These fold-coverages were used as abundances for downstream analysis.

\section{Statistical analyses and data visualization}

All statistical tests were performed in Python (v3.9.2) unless otherwise stated \parencite{PythonSoftwareFoundation.}.
Data visualizations were generated using plotly (v4.14.3) and matplotlib (v3.4.1) packages and were exported for editing in Adobe Illustrator 2020 \parencite{PlotlyTechnologiesInc..2015, Hunter.2007, AdobeInc..2020}.

\subsection{Alpha diversity}

Alpha diversity (within-group diversity using the Chao1 richness estimator) was computed using the sci-kit bio package (v0.5.6).
A Shapiro-Wilk test was used to assess the normality of Chao1 richness, followed by a one-way analysis of variance (ANOVA) for parametric data or a Kruskal-Wallis test for non-parametric data to test if differences in the Chao1 richness between treatment groups were statistically significant, as implemented by the SciPy package \parencite{PauliVirtanen.2020}.

\subsection{Beta diversity}

Beta diversity (between-group diversity) was analyzed using a principal component analysis (PCA).
A pseudocount of 0.5 was added to each feature table of abundances (taxa, genes, open reading frames) prior to center log ratio (CLR) transformation to obtain samplewise Aitchison distances --- or CLR-transformed relative abundances.
A PCA was performed on the resulting table of CLR-transformed relative abundances to investigate differences in antibiotic resistance gene, mobile genetic element, integron gene cassette open reading frame, and bacterial community composition between treatment groups using the sci-kit learn package (v0.24.1).
Permutational multivariate ANOVA (PERMANOVA) was used to determine if differences in the dispersion between treatment groups within the PCA was statistically significant using the sci-kit bio package.

\subsection{Differential abundance}

Differential abundance analysis was performed to determine if differences in the abundances of bacterial taxa, antibiotic resistance genes, mobile genetic elements, or gene cassette COG functional categories between groups was statistically significant using ANCOM-BC (v1.2.0) with Holm-Bonferroni correction as implemented in R (v4.1.0) \parencite{RCoreTeam.2021, Lin.2020}.
A one-way analysis of variance (ANOVA) was used to test if differences in the numbers of merged 16S rDNA amplicon reads between treatment groups was statistically significant as implemented by the SciPy package \parencite{PauliVirtanen.2020}.

\chapter{Results}

DNA from untreated control soil and soil exposed to a low (0.1 mg kg$^{-1}$) or high (10 mg kg$^{-1}$) dose of macrolide antibiotics was used to generate three sequence datasets:
16S rDNA was sequenced to investigate the diversity and composition of the soil bacterial community,
total metagenomic DNA was sequenced to investigate the diversity and composition of the resistome and mobilome,
and integron gene cassettes were sequenced to investigate the diversity and composition of integron gene cassette open reading frames.

\section{Sequencing statistics} \label{section:sequencing-statistics}

Bacterial 16S rDNA amplicons were sequenced to investigate differences in soil bacterial community composition and diversity in response to antibiotic exposure.
16S rDNA MiSeq sequencing generated 6.21 M reads with an average of 50.7 $\pm$ 12.8 K unique reads per sample over 12 samples  (\textit{n} = 4 for each treatment group) (Supplementary Table \ref{supp-table:sequencing-statistics-16S}).
Following quality control, 6,837 to 72,287 merged reads were used for amplicon sequence variant assignment which resulted in 2,587 amplicon sequence variants.
One control sample was excluded from the 16S rDNA amplicon analysis due to a low number of merged reads resulting from the DADA2 workflow  (\textit{n} = 2,444).
The mean numbers of merged reads were not significantly different between treatment groups after removing this sample (one-way ANOVA, $F$ = 2.3, $p$ = 0.16).
The resulting 16S rDNA sequence reads were used to establish amplicon sequence variants which were taxonomically classified, and the number of reads for each bacterial taxon were obtained.

Next, metagenomic DNA was sequenced to investigate differences in the composition and diversity of the soil bacterial community, antibiotic resistance genes, and mobile genetic elements in response to antibiotic exposure.
Metagenomic NovaSeq sequencing generated 1.49 B reads with an average of 153 $\pm$ 18.7 M unique reads per sample over nine samples (\textit{n} = 3 for each treatment group) (Supplementary Table \ref{supp-table:sequencing-statistics-metagenomic}).
The resulting metagenomic sequence reads were i) taxonomically classified to obtain a second dataset of bacterial taxa abundances, and ii) mapped to antibiotic resistance gene and mobile genetic elements to obtain abundances for these antibiotic resistance determinants.

Finally, integron gene cassette amplicons were sequenced to investigate differences in the composition and diversity of gene cassette open reading frames in response to antibiotic exposure.
Integron gene cassette HiSeq sequencing generated 9.83 Gb of reads with an average of 0.94 $\pm$ 0.18 M unique reads per sample over 24 samples (\textit{n} = 4 biological replicates, \textit{n} = 2 technical replicates) (Supplementary Table \ref{supp-table:sequencing-statistics-integron}).
These reads were assembled into 270,368 contigs which were then filtered to retain only 75,850 high-confidence gene cassettes (28\%).
These gene cassettes were predicted to contain 72,628 open reading frames of which 36,050 (48\%) were considered unique.
Integron gene cassette open reading frames were analyzed for antibiotic resistance genes at three different levels of confidence (low, moderate, high), and to assign COG functional categories.
Integron gene amplicons were mapped onto the unique open reading frames to obtain abundances for these putative protein-coding genes.

The abundances that were obtained from each of these sequencing datasets allowed us to investigate differences in soil bacterial community, antibiotic resistance gene, mobile genetic element, and integron gene cassette open reading frame composition and diversity in response to macrolide antibiotic exposure.

\section{Bacterial community composition and diversity}

The richness (one-way ANOVA $F$ = 0.78, $p$ = 0.50) and composition (PERMANOVA pseudo-$F$ = 1.0, $p$ = 0.41) of soil bacterial taxa were not significantly affected by antibiotic exposure (Supplementary Figure \ref{supp-fig:chao1-richness-taxa}; Supplementary Figure \ref{supp-fig:pca-ordplots-taxa}).
However, at the phylum level, the relative abundances of \textit{Cyanobacteria} sequence variants were decreased in the high-dosed soil (Holm-Bonferroni-adjusted $p$-value $<$ 0.001, $W$ = -5.7 $\pm$ 0.7) but not in the low-dosed soil, relative to the control.
This result was observed only in the metagenomic analysis (Figure \ref{fig:effect-sizes-sig-phyla-metagenomic}) and not in the 16S rDNA analysis (Supplementary Figure \ref{supp-fig:effect-sizes-sig-phyla-16S}).
This difference was driven by the cyanobacteria \textit{Microcoleus vaginatus} ($p$ $<$ 0.001, $W$ = -5.2 $\pm$ 0.8) and \textit{Oscillatoria nigro-viridis} ($p$ $<$ 0.001, $W$ = -5.3 $\pm$ 0.3) in the high-dosed soil.

Only three taxa were differentially abundant by over 10-fold relative to the control:
an unknown \textit{Acidobacteria Subgroup 6} sp. was increased by approximately 46-fold in the low-dosed soil ($W$ = 45.8 $\pm$ 3.8), and two unknown \textit{Chloroflexi Gitt-GS-136} spp. were increased by approximately 23-fold ($W$ = 23.0 $\pm$ 0.8) and 59-fold ($W$ = 58.6 $\pm$ 0.4) in the high-dosed soil (Table \ref{table:effect-sizes-sig-taxa}).
These differences were observed only in the 16S rDNA analysis.

Overall, five bacterial species were increased in response to antibiotic exposure (three in the low dose, two in the high dose) and six species were decreased (one in the low dose, five in the high dose) (Table \ref{table:effect-sizes-sig-taxa}).
The taxa that were differentially abundant in the low- and high-dosed soil did not overlap (were not in-common) between the 16S rDNA amplicon and metagenomic taxonomic datasets.

\begin{figure}[htpb]
	\centering
		\includegraphics[width=\textwidth]{figures/metagenomic/effect-sizes-sig-phyla-metagenomic_v02.pdf}
	\caption[Effect sizes (fold-changes) of differences in the relative abundances of bacterial phyla classified in the metagenomic analysis relative to the untreated control soil.]{
		Effect sizes (fold-changes) of differences in the relative abundances of bacterial phyla classified in the metagenomic analysis relative to the untreated control soil.
		Horizontal lines intersecting with circles are error bars, indicating the extent of Bonferroni-adjusted 95\% confidence intervals of effect sizes (\textit{n} = 3).
	}
	\label{fig:effect-sizes-sig-phyla-metagenomic}
\end{figure}

\begin{table}[htpb]
	\centering
		\includegraphics[width=\textwidth]{tables/16S/effect-sizes-sig-taxa_v03.pdf}
	\caption[Effect sizes (fold-changes) of differentially abundant soil bacterial taxa in response to macrolide antibiotic exposure at low (0.1 mg kg\textsuperscript{-1}) and high (10 mg kg\textsuperscript{-1}) doses.]{
		Effect sizes (fold-changes) of differentially abundant soil bacterial taxa in response to macrolide antibiotic exposure at low (0.1 mg kg\textsuperscript{-1}) and high (10 mg kg\textsuperscript{-1}) doses.
		Effect sizes are stated with 95\% Bonferonni-adjusted confidence intervals (CIs).
		Differential abundance analysis was performed using ANCOM-BC for the 16S rDNA analysis (16S) or metagenomic analysis (M).
		All $p$-values are Holm-Bonferroni-adjusted.
		No taxa were identified as differentially abundant by both analyses.
	}
	\label{table:effect-sizes-sig-taxa}
\end{table}

\section{Resistome and mobilome composition and diversity}

\subsection{Resistome}

A total of 583 unique antibiotic resistance genes were detected across the soil metagenomes.
High macrolide antibiotic exposure significantly increased the richness of total antibiotic resistance genes in agricultural soil (Tukey’s all-pairs test, $p$ $<$ 0.05) but no effect was observed for the low dose (Figure \ref{fig:chao1-richness}a; Supplementary Figure \ref{supp-fig:args-mges-compartment-size}).
Similarly, high exposure but not low exposure changed the composition of antibiotic resistance genes (PERMANOVA pseudo-$F$ = 1.49, $p$ $<$ 0.05, 999 permutations) (Figure \ref{fig:pca-ordplots-args-mges-integrons}a).
These differences in composition were largely driven by 21 increased antibiotic resistance genes in the high dosed soil ($p$ $<$ 0.05) (Figure \ref{fig:effect-sizes-sig-args}).
Only five antibiotic resistance genes were differentially abundant (two decreased, three increased) in the low-dosed soil and no resistance gene was differentially abundant in both treatment groups.

\begin{figure}[htpb]
	\centering
		\includegraphics[width=\textwidth]{figures/mixed/chao1-richness_v03.pdf}
	\caption[Richness (Chao1) of antibiotic resistance genes, mobile genetic elements, and integron gene cassette open reading frames.]{
		Richness (Chao1) of antibiotic resistance genes, mobile genetic elements, and integron gene cassette open reading frames.
		Richness was determined for \textbf{a}) metagenomic antibiotic resistance genes, \textbf{b}) metagenomic mobile genetic elements, and \textbf{c}) integron gene cassette open reading frames of antibiotic-exposed and -unexposed soil bacteria.
		Horizontal lines connect samples within the same treatment group for visual clarity.
		Statistically significant comparisons between the antibiotic-exposed and untreated control soil are displayed (Kruskal-Wallis test, $p$ $<$ 0.05).
	}
	\label{fig:chao1-richness}
\end{figure}

\begin{figure}[htpb]
	\centering
		\includegraphics[width=0.9\textwidth]{figures/mixed/pca-ordplots-args-mges-integrons_v04.pdf}
	\caption[PCA ordination plots (PC1, PC2) of the CLR-transformed relative abundances of \textbf{a}) metagenomic antibiotic resistance genes, \textbf{b}) metagenomic mobile genetic elements, and \textbf{c}) integron gene cassette open reading frames in antibiotic-exposed and -unexposed soil bacteria.]{
		PCA ordination plots (PC1, PC2) of the CLR-transformed relative abundances of \textbf{a}) metagenomic antibiotic resistance genes, \textbf{b}) metagenomic mobile genetic elements, and \textbf{c}) integron gene cassette open reading frames in antibiotic-exposed and -unexposed soil bacteria.
		PERMANOVA pseudo-$F$ and $p$-values with 999 permutations are displayed.
		Shaded areas correspond to 95\% confidence ellipses of treatment groups.
		Percentages of variance explained by each axis are displayed in the axis titles.
	}
	\label{fig:pca-ordplots-args-mges-integrons}
\end{figure}

\begin{figure}[htpb]
	\centering
		\includegraphics[width=0.8\textwidth]{figures/metagenomic/effect-sizes-sig-args_v07.pdf}
	\caption[\textbf{a}) Effect sizes (fold-changes) of differences in the relative abundances of antibiotic resistance genes in antibiotic-exposed soil metagenomes and \textbf{b}) their target drug classes relative to the untreated control soil (\textit{n} = 3).]{
		\textbf{a}) Effect sizes (fold-changes) of differences in the relative abundances of antibiotic resistance genes in antibiotic-exposed soil metagenomes and \textbf{b}) their target drug classes relative to the untreated control soil (\textit{n} = 3).
		\textbf{a}. Only the genes that were differentially abundant ($p$ $<$ 0.05) with an absolute effect size of at least 5 (vertical dashed bars), for either treatment group, are shown.
		Shaded circles represent genes whose abundances were significantly different from the untreated control soil and open circles represent abundances that were not significantly different.
		Horizontal lines intersecting with circles are error bars, indicating the extent of Bonferroni-adjusted 95\% confidence intervals of effect sizes.
		\textbf{b}. Light grey squares indicate that resistance to a beta-lactam antibiotic is predicted; dark grey squares indicate that resistance to a ribosome-targeting drug class is predicted; black squares indicate that resistance to a non-beta-lactam and non-ribosome-targeting drug class is predicted (other).
	}
	\label{fig:effect-sizes-sig-args}
\end{figure}

The 21 antibiotic resistance genes that had increased relative abundances in the high dosed soil were predicted to confer resistance to 11 different drug classes of antibiotics and triclosan (a biocide), especially aminoglycosides (\textit{n} = 10) and diaminopyrimidines (\textit{n} = 4) (Figure \ref{fig:increased-drug-classes}).
Sixteen of these antibiotic resistance genes were predicted to confer resistance to classes of antibiotics which, like macrolides, target the ribosome.
Only two of these increased antibiotic resistance genes were predicted to encode resistance to macrolides (\textit{mphE}, \textit{mexQ}).
The gene that had the greatest increase in relative abundance in response to high antibiotic exposure was the aminoglycoside resistance gene \textit{aph(3’’)-Ib} ($W$ = 22.9 $\pm$ 0.5, $p$ $<$ 0.05).

\begin{figure}[htpb]
	\centering
		\includegraphics[width=\textwidth]{figures/metagenomic/increased-drug-classes_v04.pdf}
	\caption[Counts of target drug classes for which resistance was predicted to be encoded by antibiotic resistance genes within the soil metagenome.]{
		Counts of target drug classes for which resistance was predicted to be encoded by antibiotic resistance genes within the soil metagenome.
		Only counts for antibiotic resistance genes that were enriched in response to macrolide antibiotic exposure are displayed ($p$ $<$ 0.05, \textit{n} = 3).
	}
	\label{fig:increased-drug-classes}
\end{figure}

Analysis of antibiotic resistance genes grouped by their target drug class indicated that the compositions of aminoglycoside, diaminopyrimidine, phenicol, tetracycline, lincosamide, and streptogramin resistance genes, but not macrolide resistance genes, were significantly altered in the high-dosed soil ($p$ $<$ 0.05, 999 permutations) (Figure \ref{fig:pca-ordplots-drug-classes}).
Of the three antibiotic resistance genes that had increased relative abundances in the low-dosed soil ($p$ $<$ 0.05), two were predicted to encode resistance to macrolide antibiotics (\textit{mexL}, $W$ = 5.6 $\pm$ 0.2; \textit{mexP}, $W$ = 5.2 $\pm$ 0.2) and one was predicted to encode resistance to aminoglycosides (\textit{aac(6’)-IIa}, $W$ = 4.0 $\pm$ 0.4). No antibiotic resistance genes had increased relative abundances in both doses.

\begin{figure}[htpb]
	\centering
		\includegraphics[width=\textwidth]{figures/metagenomic/pca-ordplots-drug-classes_v03.pdf}
	\caption[PCA ordination plots (PC1, PC2) of CLR-transformed relative abundances of antibiotic resistance genes in the untreated control soil and in the low- and high-dosed soil, grouped by their target drug class.]{
		PCA ordination plots (PC1, PC2) of CLR-transformed relative abundances of antibiotic resistance genes in the untreated control soil and in the low- and high-dosed soil, grouped by their target drug class.
		PERMANOVA pseudo-$F$ and $p$-values with 999 permutations are displayed.
		Shaded areas correspond to 95\% confidence ellipses of treatment groups.
		Percentages of variance explained by each axis are displayed in the axis titles.
	}
	\label{fig:pca-ordplots-drug-classes}
\end{figure}

Seven antibiotic resistance genes had significantly decreased relative abundances relative to the control soil: five in the high dose and two in the low dose ($p$ $<$ 0.05).
Interestingly, all seven of these resistance genes were predicted to encode beta-lactamases (Figure \ref{fig:effect-sizes-sig-args}).
\textit{bla}\textsubscript{SHV-71} ($W$ = -24.1 $\pm$ 0.3), \textit{bla}\textsubscript{SHV-165} ($W$ = -11.0 $\pm$ 0.3), \textit{bla}\textsubscript{CTX-M-117} ($W$ = -7.6 $\pm$ 0.5), \textit{E. coli} \textit{ampC} ($W$ = -5.7 $\pm$ 0.4), and \textit{bla}\textsubscript{PEDO-1} ($W$ = -4.4 $\pm$ 0.2) were decreased in the high dose, and \textit{bla}\textsubscript{TEM-1} ($W$ = -4.9 $\pm$ 0.3) and \textit{bla}\textsubscript{TEM-22} ($W$ = -4.1 $\pm$ 0.8) were decreased in the low dose.

\subsection{Mobilome}

In addition to antibiotic resistance genes, the composition and diversity of mobile genetic elements within the soil metagenome was investigated.
Overall, 398 unique mobile genetic element variants were detected across the soil metagenomes, including several transposases and insertion sequence elements (e.g IS91, IS26).
As observed with antibiotic resistance genes, the richness of mobile genetic elements was significantly increased in the soil metagenome (Tukey’s all-pairs test, $p$ $<$ 0.05) (Figure \ref{fig:chao1-richness}b; Supplementary Figure \ref{supp-fig:args-mges-compartment-size}), and the composition of mobile genetic elements was significantly affected by the high dose of macrolides (PERMANOVA pseudo-$F$ = 2.01, $p$ $<$ 0.05, 999 permutations) (Figure \ref{fig:pca-ordplots-args-mges-integrons}b).

This altered composition of mobile genetic elements in the high-dosed soil was largely driven by 23 mobile genetic element variants with increased relative abundances ($p$ $<$ 0.05) (Figure \ref{fig:effect-sizes-sig-mges}).
Of these 23 increased mobile genetic elements, 15 were identified as \textit{tnpA}, three as \textit{intI1}, three as \textit{qacE$\Delta$1}, one as IS91, and one as \textit{tnpAN} variants.
The maximum effect size of the mobile genetic element variants that were increased in the high dose was $W$ = 23.8 $\pm$ 0.1 for \textit{intI1} ($p$ $<$ 0.05).

\begin{figure}[htpb]
	\centering
		\includegraphics[width=\textwidth]{figures/metagenomic/effect-sizes-sig-mges_v03.pdf}
	\caption[Effect sizes (fold-changes) of differences in the relative abundances of mobile genetic elements (MGEs) in antibiotic-exposed soil metagenomes relative to the untreated control soil (\textit{n} = 3).]{
		Effect sizes (fold-changes) of differences in the relative abundances of mobile genetic elements (MGEs) in antibiotic-exposed soil metagenomes relative to the untreated control soil (\textit{n} = 3).
		Only the mobile genetic elements that were differentially abundant ($p$ $<$ 0.05) with an absolute effect size of at least 5 (vertical dashed bars), for either treatment group, are shown.
		The name of the moobile genetic element is shown on the left, and the GenBank accession number of the reference sequence's genome is shown on the right.
		Shaded circles represent mobile genetic elements whose abundances were significantly different from the untreated control soil and open circles represent abundances that were not significantly different.
		Horizontal lines intersecting with circles are error bars, indicating the extent of Bonferroni-adjusted 95\% confidence intervals of effect sizes.
	}
	\label{fig:effect-sizes-sig-mges}
\end{figure}

The only mobile genetic element variant with an increased relative abundance in the low-dosed soil was identified as \textit{tnpA} ($W$ = 6.0 $\pm$ 0.3, $p$ $<$ 0.05) (Figure \ref{fig:effect-sizes-sig-mges}).
Of the three mobile genetic element variants that were decreased in the low-dosed soil (IS91, \textit{n} = 2; \textit{tnpA}, \textit{n} = 1), one IS91 variant was similarly decreased in the high-dosed soil (low dose, $W$ = -5.8 $\pm$ 0.4; high dose, $W$ = -5.7 $\pm$ 0.4).
No other mobile genetic element variants were differentially abundant in both doses.

\section{Composition and diversity of open reading frames from integron gene cassettes}

Both the richness (one-way ANOVA $F$ = 0.05, $p$ = 0.95) and composition (PERMANOVA pseudo-$F$ = 1.0, $p$ = 0.42) of integron gene cassette open reading frames were unaffected by antibiotic exposure (Figure \ref{fig:chao1-richness}c; Figure \ref{fig:pca-ordplots-args-mges-integrons}c).
Overall, 370 open reading frames (1\%) were identified as differentially abundant relative to the untreated control soil ($p$ $<$ 0.05), and of these 370 open reading frames, more were differentially abundant in the high-dosed soil (\textit{n} = 246, 67\%) than the low-dosed soil (\textit{n} = 144, 39\%).

\begin{figure}[htpb]
	\centering
		\includegraphics[width=0.85\textwidth]{figures/integron/effect-sizes-sig-integron-orfs_v04.pdf}
	\caption[Effect sizes (fold-changes) of differences in the relative abundances of integron gene cassette open reading frames in antibiotic-exposed soil bacteria relative to the untreated control soil.]{
		Effect sizes (fold-changes) of differences in the relative abundances of integron gene cassette open reading frames in antibiotic-exposed soil bacteria relative to the untreated control soil.
		Only the open reading frames that were differentially abundant ($p$ $<$ 0.05, \textit{n} = 3) with an absolute effect size $\ge$ 10 (vertical dashed bars), for either treatment group, are shown.
		The open reading frame (ORF) ID is shown on the left and the assigned COG functional category on the inner-right (if available).
		The name and highest confidence level (low, moderate, high) of predicted antibiotic resistance genes are shown on the outer-right.
		Shaded circles represent open reading frames whose abundances were significantly different from the untreated control soil and open circles represent abundances that were not significantly different.
		Horizontal lines intersecting with circles are error bars, indicating the extent of Bonferroni-adjusted 95\% confidence intervals of effect sizes.
		COG functional categories are as follows:
		K = Transcription,
		O = Post-translational modification, protein turnoover, chaperones;
		S = Function unknown.
	}
	\label{fig:effect-sizes-sig-integron-orfs}
\end{figure}

In total, 60 to 2,997 unique open reading frames (0.2 to 8.4\%) were predicted to encode antibiotic resistance depending on the confidence level used (see Chapter \ref{section:cassette-sequence-analysis}).
For the antibiotic resistance genes predicted at each confidence level, the most frequently detected target drug class of antibiotic resistance genes was aminoglycoside and the most frequently detected drug resistance mechanism was antibiotic inactivation.
Depending on the confidence level, 1 to 17 putative antibiotic resistance genes had increased relative abundances in response to antibiotic exposure and 1 to 13 putative antibiotic resistance genes decreased ($p$ $<$ 0.05) (Supplementary Table \ref{supp-table:integron-args-drug-classes}).
However, no putative antibiotic resistance genes were increased or decreased in both treatment groups at any confidence level.

Integron gene cassette open reading frames were also assigned COG functional categories to investigate if macrolide exposure changed the overall function of the cassette metagenome.
Only 5,206 (15\%) unique open reading frames could be assigned a functional category, and of those, 2,053 (39\%) were assigned a functional category other than 'function unknown' (S) (Supplementary Figure \ref{supp-fig:percentages-cog-categories}).
The open reading frames that were assigned to functional category EK (E: amino acid transport and metabolism; K: transcription) had slightly increased relative abundances ($W$ = 3.4 $\pm$ 1.1, $p$ $<$ 0.05), and those assigned to category DJ (D: cell cycle control, mitosis and meiosis; J: translation, ribosomal structure and biogenesis) had slightly decreased relative abundances ($W$ = -3.6 $\pm$ 0.8, $p$ $<$ 0.05) in the soil bacteria exposed to a high dose of macrolide antibiotics, but only a few open reading frames were assigned to each of these categories (EK, \textit{n} = 3; DJ, \textit{n} = 2).

\chapter{Discussion}

The purpose of this project was to investigate the effect of long-term macrolide antibiotic exposure on the soil bacterial community, resistome, and mobilome --- and more specifically, to determine if an environmentally realistic dose of antibiotics for a biosolids exposure scenario could promote clinically relevant antibiotic resistance in soil bacteria.
Human and environmental health are interconnected as described by the One Health framework, and antibiotic resistance in soil bacteria may affect antibiotic resistance in the human microbiome or in pathogens.

\section{Realistic antibiotic exposure does not affect the diversity or composition of the soil bacterial community, resistome, or mobilome}

Overall, no effect of environmentally realistic antibiotic exposure (low dose, 0.1 mg kg\textsuperscript{-1}) on the diversity or composition of the soil bacterial community, resistome, or mobilome was detected.
This dose is similar to what would be expected in soil following the land-application (1--10\% dw dw\textsuperscript{-1}) of municipal biosolids containing 95\textsuperscript{th} percentile concentrations of erythromycin, clarithromycin, and azithromycin antibiotics.
The absence of a treatment effect for any of these endpoints indicates that repeated annual application of biosolids in agriculture is unlikely to promote antibiotic resistance in agricultural soil at levels that would be of concern to human health.

However, three antibiotic resistance genes had increased relative abundances in response to antibiotic exposure at the low dose, and two of them are known to be associated with resistance to macrolide antibiotics (Figure \ref{fig:effect-sizes-sig-args}).
\textit{mexL} and \textit{mexP} are members of the \textit{mex} gene family which are components of chromosomally encoded efflux pumps in \textit{Pseudomonas} spp. \parencite{Mima.2005, Chuanchuen.2005}.
The \textit{mexL} gene encodes a repressor for \textit{mexJK} transcription, which are members of the efflux pump-encoding \textit{mexJK-OpmH} (triclosan resistance) and \textit{mexOprM} (macrolide, tetracycline resistance) operons \parencite{Chuanchuen.2005}.
The \textit{mexP} gene encodes the membrane fusion protein for the multi-drug efflux pump that is encoded by the \textit{MexPQ-OpmE} operon, which is known to confer resistance to several antibiotic drug classes including macrolides \parencite{Mima.2005}.
The other antibiotic resistance gene that was increased in the low dosed-soil was \textit{aac(6')-IIa}, which encodes an aminoglycoside acetyltransferase that is distributed among a variety of gram-negative pathogens, including \textit{Pseudomonas} spp., and is carried by plasmids and integrons \parencite{Shaw.1989, Partridge.2009}.
Only one mobile genetic element variant, \textit{tnpA}, was increased in the low-dosed soil (Figure \ref{fig:effect-sizes-sig-mges}).
The \textit{tnpA} gene, of which there are many variants, encodes a transposase which is involved in the mobilization of several bacterial mobile genetic elements, such as transposons and insertion sequences across a broad host range \parencite{Partridge.2018}.
Three bacterial taxa were increased and are discussed in Chapter \ref{section:fastidious-taxa}.

Despite the increased relative abundances of three antibiotic resistance genes, one mobile genetic element variant, and three bacterial taxa in the low-dosed soil --- the overall diversity and composition of the soil bacterial community, resistome, and mobilome was unchanged by the low dose as indicated by richness (Figure \ref{fig:chao1-richness}, Supplementary Figure \ref{supp-fig:chao1-richness-taxa}) and PCA ordination plots (Figure \ref{fig:pca-ordplots-args-mges-integrons}, Supplementary Figure \ref{supp-fig:pca-ordplots-taxa}).
The increased abundances of a few antibiotic resistance genes in the low-dosed soil and that of \textit{tnpA} could be due to the increased abundance of a particular taxon that is intrinsically resistant to macrolide antibiotics, such as the human pathogen \textit{Pseudomonas aeruginosa}, though no taxa that were both significantly increased and known to carry this assortment of genes were detected.
These results are in clear contrast to the high dose of macrolide antibiotics.

\section{Unrealistically high antibiotic exposure alters the diversity and composition of the soil bacterial resistome and mobilome}

At a high dose of macrolide antibiotics (10 mg kg\textsuperscript{-1}) --- approximately 100-fold greater than the concentrations of macrolides that would be expected to result from land-application of municipal biosolids --- significant effects on the diversity and composition of the soil bacterial resistome and mobilome were detected.
While this dose is considered unrealistically high for soil receiving biosolids with macrolide concentrations within an upper-realistic range (upper 95\textsuperscript{th} percentile), the maximum azithromycin concentration detected in the \cite{U.S.EnvironmentalProtectionAgency.2009} biosolids survey (5 mg kg\textsuperscript{-1}) was within an order-of-magnitude of the high dose used in this present study.
This high dose of macrolide antibiotics has also been detected in sediments surrounding macrolide antibiotic manufacturing facilities \parencite{GonzalezPlaza.2019}.
Therefore, while this dose is described as unrealistically high for a biosolids land-application scenario, comparable doses are certainly seen under other environmental contexts, and the findings of this experiment may be useful for predicting antibiotic resistance in other macrolide-contaminated environments.

The high dose of macrolide antibiotics significantly increased the number of unique antibiotic resistance genes that were detected within the soil metagenome (Figure \ref{fig:chao1-richness}a).
The majority of antibiotic resistance genes that were detected in the metagenome were detected within all three groups (control, low, and high), but approximately twice as many unique antibiotic resistance genes were detected within the high treatment group alone than the control and low groups alone (Supplementary Figure \ref{supp-fig:args-mges-compartment-size}a).
The most likely explanation for the increased number of unique antibiotic resistance genes in the high-dosed soil is selection or co-selection for genes that were below the detection limit for the control and low groups, and were raised above the limit of detection by high antibiotic exposure (but not significantly increased based upon the differential abundance analysis).

In addition, a high dose of macrolide antibiotics changed the composition of antibiotic resistance genes within the soil metagenome (Figure \ref{fig:pca-ordplots-args-mges-integrons}a), and this effect extended to when antibiotic resistance genes were grouped by several drug classes (Figure \ref{fig:pca-ordplots-drug-classes}).
These differences in composition indicated that the relative abundances of several antibiotic resistance genes and target drug classes were more similar within the high-dosed soil than in the control and low-dosed soil.
The altered composition of antibiotic resistance genes in the high-dosed soil was driven by increased relative abundances of 21 antibiotic resistance genes in the high dose --- only two of which are known to confer resistance to macrolide antibiotics, while 11 are known to confer resistance to non-macrolide, ribosome-targeting drug classes (aminoglycoside, phenicol, tetracycline) (Figure \ref{fig:effect-sizes-sig-args}).

Like macrolides, the aminoglycoside, phenicol, and tetracycline drug classes of antibiotics also target the bacterial ribosome, but bind to different locations within the ribosome than do the macrolides \parencite{Pyorala.2014, Lohsen.2019}.
Because the precise ribosomal targets of these drug classes are different to that of macrolides, cross-resistance of antibiotic resistance genes to macrolides and drug classes other than lincosamide and streptogramin B antibiotics (whose targets overlap with that of macrolides) is uncommon, except for antibiotic resistance genes that encode multi-drug efflux pumps such as the \textit{mex} gene family in \textit{Pseudomonas} spp.
The increased relative abundances of non-macrolide antibiotic resistance genes in the high-dosed soil strongly suggests co-selection via co-resistance rather than cross-resistance
(except for \textit{mexQ}), which could be facilitated by mobile genetic elements.

Co-resistance to different drug classes of antibiotics can occur due to the genetic linkage of antibiotic resistance on mobile genetic elements such as class 1 integrons, or when antibiotic resistance genes are carried within the same host \parencite{Pal.2015}.
Of the 21 increased antibiotic resistance genes in the unrealistically high-dosed soil metagenome, eight are known to be associated with class 1 integrons (\textit{sul1}, \textit{aac(3)-Ib}, \textit{aadA}, \textit{aadA15}, \textit{aadA22}, \textit{aadA24}, \textit{dfrA17}, \textit{dfrA15}), which agrees with the increased relative abundance of \textit{intI1} and \textit{qacE$\Delta$1} (a component of the 3' conserved sequence) in the high dose \parencite{Partridge.2009, Yan.2006, Herrero.2008}, and with previously reported quantitative PCR observations \parencite{Lau.2020}.
All of these antibiotic resistance genes have been detected in gram-negative human pathogens \parencite{Alcock.2020}.

Despite the increased relative abundances of several metagenomic antibiotic resistance genes that are known to be associated with gene cassettes (except \textit{sul1}, which is a member of the 3' conserved sequence), integron gene cassette richness (Figure \ref{fig:chao1-richness}c) and composition (Figure \ref{fig:pca-ordplots-args-mges-integrons}c) were unaffected by antibiotic exposure, as determined by the integron gene cassette sequence analysis.
The absence of a treatment effect of macrolide antibiotic exposure on gene cassette richness and composition may be due to how \textit{intI1} gene expression --- and therefore, gene cassette recombination --- is regulated in bacteria.

The transcription of \textit{intI1} is regulated by the bacterial SOS system which is "a coordinated response to DNA damage" that is present in most bacteria \parencite{Maslowska.2019}.
Antibiotic drug classes that damage DNA (e.g. fluoroquinolones, nitrofurans) or affect DNA synthesis (e.g. sulfonamides, trimethoprim) induce the bacterial SOS response, which increases the expression of \textit{intI1} and thereby triggers integron gene cassette recombination \parencite{Guerin.2009, Baharoglu.2010}.
Antibiotic drug classes that do not damage DNA, such as macrolides, likely do not induce the bacterial SOS response and therefore do not increase the expression of \textit{intI1}, which would otherwise trigger integron gene cassette recombination \parencite{Hastings.2004}.
Therefore, macrolide antibiotic exposure of soil bacteria may not trigger gene cassette recombination, but antibiotics that induce the SOS response in soil bacteria may alter the richness or composition of the gene cassette metagenome and should be investigated for these effects.

Alternatively, because environmental integron gene cassettes were sequenced and class 1 integrons were not specifically targeted in the integron gene cassette targeted amplicon sequencing, the environmental classes of integrons (of which there are hundreds) may have overwhelmed our gene cassette sequencing dataset, leaving few reads for class 1 integron gene cassettes, whose diversity and composition may have been affected by antibiotic exposure.
A future study investigating the response of the gene cassette metagenome of class 1 integrons to macrolide antibiotics could reveal differences that our compositional data analysis was not powered to detect.

Of the remaining non-cassette-associated metagenomic antibiotic resistance genes that were increased in the high-dosed soil, five are known to be carried on plasmids in human pathogens (\textit{sul2}, \textit{aac(6')-Ib7}, \textit{pp-flo}, \textit{mphE}, \textit{ant(3'')-IIa}), \textit{tet(33)} is carried by the insertion sequence IS6100, and \textit{aph(3’’)-Ib} is carried on several mobile genetic elements including plasmids and transposons \parencite{Alcock.2020, Tauch.2002}.
The remaining genes with increased relative abundances are known to be chromosomally-encoded in \textit{Pseudomonas} spp. (\textit{aph(3’)-IIb}, \textit{oprN}, \textit{mexH}, \textit{triC}, \textit{mexQ}) or in \textit{Burkholderia} spp. (\textit{opcM}) \parencite{Hachler.1996, Mesaros.2007, Mima.2005, Mima.2007, Burns.1996}.
All of these remaining genes with the exception of \textit{aph(3')-IIb}, an aminoglycoside phosphotransferase, encode components of antibiotic efflux pumps.

Overall, of the 21 increased antibiotic resistance genes in the high-dosed soil, 15 are known to be carried by mobile genetic elements and all are known to be associated with human pathogens.
In considering the threat of these antibiotic resistance genes to human health, future research should be performed to determine if these genes reside within pathogenic bacteria in the soil or have the potential to be mobilized to human pathogens.

The high dose of antibiotics similarly increased the number of unique mobile genetic element variants (Figure \ref{fig:chao1-richness}b) and altered the composition of the mobilome (Figure \ref{fig:pca-ordplots-args-mges-integrons}).
More mobile genetic element variants were detected in the high-dosed soil group alone (\textit{n} = 119) than were shared between any combination of the other groups, suggesting that the high dose of macrolides raised many mobile genetic element variants over the limit of detection (Supplementary Figure \ref{supp-fig:args-mges-compartment-size}b).
Of the 23 mobile genetic element variants with increased relative abundances in the high-dosed soil (Figure \ref{fig:effect-sizes-sig-mges}), most were \textit{tnpA} variants (\textit{n} = 15), which suggests that some of the increased antibiotic resistance genes may have been mobilized by transposons or insertion sequences.

The exact mechanism of co-selection of macrolide and non-macrolide antibiotic resistance genes in the high-dosed soil could not be elucidated.
However, because antibiotic resistance genes that are known to be associated with several types of mobile genetic elements were increased, and several mobile genetic element variants were increased, it's plausible that multiple co-selection processes were active in the high-dosed soil simultaneously.

\section{Antibiotic exposure enriches for fastidious taxa} \label{section:fastidious-taxa}

In this study, increased relative abundances of three bacterial taxa in the low-dosed soil and two taxa in high-dosed soil were detected (Table \ref{table:effect-sizes-sig-taxa}).
For the low-dosed soil, the effect sizes for two of the three increased taxa were relatively low ($W$ $<$ 5), but an unknown \textit{Acidobacteria Subgroup 6} taxon was over 45-fold more abundant in the low-dosed soil than in the control.
This taxon was present in both antibiotic-treated groups but not in the control soil.

Acidobacteria are largely uncultivated, highly abundant bacteria in agricultural soil and play an important role in shaping the soil bacterial community through their decomposition of organic carbon \parencite{Solden.2016, Banerjee.2016, Banerjee.2016b}.
Furthermore, acidobacteria are a known reservoir of macrolide antibiotic resistance in urban surface waters through their expression of the \textit{erm} gene family, and have been reported to be increased in macrolide-polluted sediments, suggesting intrinsic macrolide resistance among some taxa \parencite{Yi.2019, Milakovic.2020}.
Conservatively, the unknown \textit{Acidobacteria Subgroup 6} taxon may represent a macrolide-resistant decomposer, but more speculatively, could represent an organism that is able to use macrolides as an alternative source of carbon.
Further studies would be required to investigate the macrolide biodegradation potential of this taxon.

For the high-dosed soil, the effect sizes for both increased taxa were high ($W$ $>$ 20) and both were identified as unknown \textit{Chloroflexi Gitt-GS-136} spp. (Table \ref{table:effect-sizes-sig-taxa}).
Chloroflexi are fastidious bacteria with diverse metabolisms and, like acidobacteria, are a known reservoir of macrolide resistance in the environment, though their overall role in environmental antibiotic resistance is still poorly understood \parencite{Gupta.2013, Islam.2019, Yi.2019}.
To our knowledge, this phylum has not been reported to carry any of the antibiotic resistance genes that were increased in this study, though the relationship between \textit{Chloroflexi} taxa and antibiotic resistance remains understudied \parencite{Razavi.2017}.

In the present study, it was assumed that the bacterial hosts of the antibiotic resistance genes that were increased at the high dose would be revealed as differentially abundant in at least one of the taxonomic analyses, but it's possible that the bacterial taxa that hosted these resistance genes were not significantly differentially abundant, yet were still sufficiently increased to enrich for antibiotic resistance.
These bacterial taxa could be revealed in a future co-abundance network analysis to identify taxa whose relative abundances are correlated with those of antibiotic resistance genes and mobile genetic elements, thereby allowing us to identify candidate taxa as hosts of these gene targets \parencite{Forsberg.2014}.

Although there was no significant effect of macrolide antibiotic exposure at either dose on the overall richness or composition of the soil bacterial community, some taxa are known to respond to exposure:
in a previous investigation of the persistence of macrolide antibiotics in soils that were annually exposed to a low or high dose of erythromycin, clarithromycin, and azithromycin for five years, or were left untreated, macrolide antibiotics were degraded more rapidly in the soils with an exposure history to macrolides than in the untreated control soil \parencite{Topp.2016}.
It is possible that \textit{Acidobacteria} or \textit{Chloroflexi} taxa may have played a role in the accelerated biodegradation of these macrolide antibiotics.

\section{Unrealistically high antibiotic exposure decreases relative abundances of cyanobacteria}

The only bacterial phylum that was differentially abundant in response to antibiotic exposure was \textit{Cyanobacteria} (Figure \ref{fig:effect-sizes-sig-phyla-metagenomic}).
The relative abundances of \textit{Cyanobacteria} sequence variants were decreased in the high-dosed soil but not in the low-dosed soil, and this effect was observed only in the metagenomic analysis and not in the 16S rDNA analysis (Supplementary Figure \ref{supp-fig:effect-sizes-sig-phyla-16S}).

Cyanobacteria have recently been considered as indicator species for antibiotic pollution of aquatic ecosystems due to their sensitivity to several drug classes of antibiotics \parencite{CommitteeforMedicinalProductsforHumanUse.2015, LePage.2017}, but this response is not uniform across all species and to all antibiotics \parencite{LePage.2017, Dias.2015}.
For example, the MIC of the cyanotoxin-producing cyanobacterium \textit{Microcystis aeruginosa} to $\beta$-lactam antibiotics can be as low as 0.1 mg L\textsuperscript{-1}, while the MICs of $\beta$-lactams for the tropical cyanobacteria \textit{Gloeocapsa} sp. and \textit{Chroococcidiopsis} sp. may be 100-fold greater \parencite{Dias.2015, Reynaud.1986}.

The decreased relative abundance of \textit{Cyanobacteria} sequence variants in the high-dosed soil of this present study suggests that the high dose, but not the low dose of macrolides is inhibitory to at least some cyanobacteria.
The minimum NOECs of azithromycin and erythromycin for growth inhibition of cyanobacteria were reported to be at-most 0.0015 and 0.0062 mg L\textsuperscript{-1}, which are approximately 20--70-fold lower than the concentration of macrolides in the low-dosed soil, and 1,600--6,700-fold lower than the concentration of macrolides in the high-dosed soil \parencite{LePage.2019}.
Therefore, the decreased abundance of \textit{Cyanobacteria} sequence variants in the high-dosed soil of this present study is in agreement with the known NOECs for erythromycin and azithromycin, but it should be noted that these values were determined for an aqueous environment and are likely different for soil.
Our inability to detect an effect of macrolide antibiotic exposure at the low dose may be due to a higher MIC for soil cyanobacteria or insufficient sensitivity to detect this effect using metagenomic sequencing.

The detection of this treatment effect in the high-dosed soil for only one of two taxonomic analyses may be due to differences in how abundances are calculated for each approach:
for the metagenomic analysis, metagenomic sequence reads were matched to a database of clade-specific marker genes and fold-coverages of these genes were obtained;
for the 16S rDNA analysis, amplicon sequence variants of 16S rDNA sequences were constructed and assigned taxonomy using a 16S rRNA gene database, and the number of times each variant was observed was counted.
Because 16S rRNA gene copy numbers are variable in bacteria, the relative abundances obtained from the 16S rDNA analysis are biased towards bacterial genomes with high copy numbers of the 16S rRNA gene \parencite{Kembel.2012}, whereas the metagenomic sequence analysis excluded the use of multicopy marker genes to assign taxonomy for this reason \parencite{Segata.2012}.

Another potential concern with the results obtained from the 16S rDNA sequencing experiment was the low number of merged reads resulting from the DADA2 workflow (Supplementary Table \ref{supp-table:sequencing-statistics-16S}).
A low number of merged reads can result from poor sequence quality or from excessive trimming of the 3’ ends of paired-end reads.
Our quality control analysis revealed overall good sequence quality for the 16S rDNA dataset, thus it is likely that the Trimmomatic parameters that were used need to be re-adjusted to optimize the read-merging step while also discarding low-quality bases.
This loss of data could explain why \textit{Cyanobacteria} sequence variants were not identified as differentially abundant in the 16S rDNA sequence analysis but were identified as differentially abundant in the metagenomic sequence analysis.

\section{High antibiotic exposure decreases relative abundances of $\beta$-lactam resistance genes}

Of the seven antibiotic resistance genes that were decreased in response to macrolide antibiotic exposure (five in the high dose, two in the low dose), all were predicted to encode resistance to $\beta$-lactam antibiotics (Figure \ref{fig:effect-sizes-sig-args}).
$\beta$-lactam antibiotics are bactericidal against both gram-negative and gram-positive bacteria by inhibiting synthesis of the cell wall, thereby leading to lysis and cell death \parencite{Balsalobre.2019}.
The $\beta$-lactam drug class of antibiotics was among the first to be brought to the drug market with the discovery of penicillin in 1928 by Alexander Fleming \parencite{Fleming.1929}.
The subsequent industrialized production and mass consumption of penicillins by the mid-1940's has resulted in increased acquired resistance to $\beta$-lactams, especially due to methicillin-resistant strains of \textit{Staphylococcus aureus} \parencite{PublicHealthAgencyofCanada.2020}.

$\beta$-lactam resistance genes are highly abundant in soil bacteria, even in the absence of anthropogenic antibiotic pollution, and over 90\% of these genes are encoded chromosomally \parencite{Dunivin.2019, vanGoethem.2018, Mindlin.2017}.
Of the $\beta$-lactam resistance genes that were decreased in relative abundance, two SHV-family $\beta$-lactamase encoding genes (\textit{bla}\textsubscript{SHV-71}, \textit{bla}\textsubscript{SHV-165}), one CTX-M $\beta$-lactamase (\textit{bla}\textsubscript{CTX-M-117}), one PEDO-family metallo-$\beta$-lactamase (\textit{bla}\textsubscript{PEDO-1}), and one \textit{ampC}-type $\beta$-lactamase (\textit{E. coli ampC}) were decreased in the high-dosed soil, while two TEM-family $\beta$-lactamase encoding genes (\textit{bla}\textsubscript{TEM-1}, \textit{bla}\textsubscript{TEM-22}) were decreased in the low-dosed soil.
\textit{bla}\textsubscript{TEM-1} was the first plasmid-associated $\beta$-lactam resistance gene to be identified and has since spread throughout gram-negative pathogens (e.g. \textit{Acinetobacter baumanii}, \textit{E. coli}, \textit{Klebsiella pneumoniae}).
Other members of the TEM-family of $\beta$-lactamase genes, including \textit{bla}\textsubscript{TEM-22}, have a more narrow host range but confer resistance to extended-spectrum $\beta$-lactams (able to hydrolyze oximino-cephalosporins) \parencite{Bradford.2001, Garlet.1993}.

The most likely explanation for the decreased abundances of $\beta$-lactam resistance genes in the macrolide antibiotic-exposed soil is the decreased abundance of macrolide-susceptible bacteria carrying these resistance genes.
While the co-selection of several non-macrolide antibiotic resistance genes in the high-dosed soil may have been due to genetic linkage between macrolide and non-macrolide antibiotic resistance genes, it is possible that macrolide and $\beta$-lactam resistance genes were infrequently genetically linked in the soil bacteria that were sampled, and that the $\beta$-lactam-resistant bacteria were outcompeted by macrolide-resistant bacteria in the presence of macrolides.
None of the decreased taxa in this study (\textit{Arthrobacter globiformi}, \textit{Arthobacter} sp. Leaf69, \textit{Mycolicibacterium tusciae}, \textit{M. vaginatus}, \textit{O. nigro-viridis}, \textit{Ramlibacter sp. Leaf400}) are known to carry $\beta$-lactam resistance genes, although one $\beta$-lactam resistance gene \textit{estA} has been identified in \textit{Arthrobacter nitroguajacolicus} Rü61a and several have been identified in the plasmidome of \textit{Mycolicibacterium} spp.

\section{Policy implications}

There are currently no globally accepted standards for setting limits for pollutant levels in biosolids.
In Europe, a “precautionary principle”-based approach for managing pollutant concentrations in biosolids and biosolids-applied soils has been used;
the absence of toxicity and fate data for these pollutants has led to huge variability in limits for pollutants in biosolids-applied soils between individual European nations --- sometimes by 2–3 orders of magnitude \parencite{McCarthy.2015}.
In the United States, the United Kingdom, and Canada, a risk assessment-based approach for managing pollutant concentrations in biosolids has been preferred \parencite{McCarthy.2015}.
In the United States, limits for pollutant levels have been established by considering different exposure pathways to humans (e.g. crop consumption, groundwater contamination), and by using the conservative approach of considering risk to the most highly-exposed individuals in society \parencite{McCarthy.2015}.

In Canada, most provinces and territories have set limits for pollutants (mostly inorganic) in biosolids, though this has not always been the case:
as more data of pollutant concentrations in biosolids has become available, more pollutants have been added to lists of agents requiring further research --- including antibiotics \parencite{Sabourin.2012, U.S.EnvironmentalProtectionAgency.2009, WaterEnvironmentAssociationofOntario.2010}.
Today, only two provinces in Canada (Québec and Nova Scotia) have set maximum limits for levels of organic compounds in biosolids, and no provinces or territories have set limits for antibiotics \parencite{McCarthy.2015}.

Due to the absence of experimental data on the effects of antibiotics in biosolids on soil bacteria, risk assessments based upon PNECs have been performed to guide policy decisions for managing antibiotic concentrations in biosolids \parencite{Jensen.2012, Eriksen.2009}.
Unfortunately, these risk assessments have faced many challenges, including the absence of existing toxicity and fate data for antibiotics (especially in biosolids-amended soil), and a failure to consider the simultaneous exposure to other antibiotics (and other pollutants) present in biosolids \parencite{McCarthy.2015}.
Generally, environmental risk assessments of antibiotics do not measure the potential for selection of antibiotic resistance \parencite{Lee.2019}.
However, a risk assessment by the Norwegian Scientific Committee for Food Safety compared the predicted environmental concentrations of 37 antibiotics in biosolids-applied soil to the MIC values of these antibiotics for \textit{Escherichia coli} and \textit{E. faecium};
they concluded that antibiotic resistance is unlikely to be promoted at an application rate of 60 tons biosolids ha \textsuperscript{-1} soil for all tested antibiotics except for the fluoroquinolone antibiotic, ciprofloxacin \parencite{Eriksen.2009}.
However, \textit{E. coli} and \textit{E. faecium} are intrinsically resistant to many of the antibiotics that were tested in this risk assessment, including erythromycin, clarithromycin, and azithromycin, and therefore cannot be used to conclude that antibiotic resistance will not be selected for in other biosolids-exposed soil bacteria.

In this thesis, I investigated the effects of long-term, repeated exposure of macrolide antibiotics on the soil bacterial community, resistome, and mobilome using sequencing-based methods, and I determined that an environmentally realistic dose of macrolides for a biosolids exposure scenario is unlikely to change soil bacterial diversity or promote clinically relevant antibiotic resistance.
These results suggest that typical concentrations of macrolide antibiotics within Canadian municipal biosolids are unlikely to harm environmental and human health in the context of antibiotic resistance.
However, the absence of an intermediate concentration between the low dose (0.1 mg kg\textsuperscript{-1}) and the high dose (10 mg kg\textsuperscript{-1}) means that I was unable to precisely determine the 'threshold concentration' beyond which the soil resistome and mobilome were significantly affected by macrolide antibiotic exposure for the soils used in this study.
If this threshold concentration were to be within the range of 0.1 to 1 mg kg\textsuperscript{-1}, there could be cause-for-concern for some biosolids with a high macrolide antibiotic load to promote antibiotic resistance in soil.
Furthermore, an intermediate concentration of macrolides (1 mg L\textsuperscript{-1}) is more likely to be observed in other anthropogenically polluted environments than the high dose \parencite{Bielen.2017}.

Overall, to protect human and environmental health, more data are required to establish acceptable limits for antibiotics in biosolids that are intended for agricultural use.
A similar investigation to this present study at intermediate concentrations of macrolides and in different contaminated environments may reveal similar treatment effects to those observed in the high-dosed bacteria of this study.
Future research is needed to elucidate the range of concentrations within which the soil resistome and mobilome are affected by macrolide antibiotic exposure.


% Add a line for bibliography in ToC
\addcontentsline{toc}{chapter}{Bibliography}

\todo[color=blue!20]{
  I'll go through all of the references on the following pages and make sure formatting/names are correct after I'm finished adding all of the citations to the thesis.
  Is the overall format okay?
  Should I include more/less author names?
  URLs?
  DOIs?
}
\todo[color=blue!20]{For the Committee for Medicinal Products for Human Use (2006) reference, the report was initially published in 2006 but updated in 2015. Should I indicate the year as 2006 or 2015?}
\printbibliography

% Appendices.
\begin{appendices}
\chapter{Supplementary Information}
\myappendices{Supplementary Information}

%-------------------------------------------------------------------------------
% Supplementary Figures
%-------------------------------------------------------------------------------

\begin{suppfigure}[H]
	\centering
		\includegraphics[width=\textwidth]{supp-figures/16S/effect-sizes-sig-phyla-16S_v01.pdf}
	\caption[Effect sizes of bacterial phyla classified in the metagenomic analysis.]{
		Effect sizes (fold-changes) of differences in the relative abundances of bacterial phyla classified in the 16S rDNA analysis relative to the untreated control soil (\textit{n} = 4 for antibiotic-exposed groups, \textit{n} = 3 for untreated control group).
		Horizontal lines intersecting with circles are error bars, indicating the extent of Bonferroni-adjusted 95\% confidence intervals of effect sizes.
	}
	\label{supp-fig:effect-sizes-sig-phyla-16S}
\end{suppfigure}

\begin{suppfigure}[H]
	\centering
		\includegraphics[width=0.8\textwidth]{supp-figures/mixed/chao1-richness-taxa_v02.pdf}
	\caption[Richness of bacterial taxa.]{
		Richness (Chao1) of bacterial taxa as classified by \textbf{a}) the 16S rDNA analysis or \textbf{b}) the metagenomic analysis in macrolide antibiotic-exposed and untreated control soil.
		There were no statistically significant differences between the antibiotic-exposed and -unexposed groups.
		\textsuperscript{\Cross} \textit{n} = 4 for the antibiotic-exposed groups, \textit{n} = 3 for the untreated control group.
	}
	\label{fig:chao1-richness-taxa}
\end{suppfigure}

\begin{suppfigure}[htpb]
	\centering
		\includegraphics[width=0.75\textwidth]{supp-figures/mixed/pca-ordplots-taxa-metagenome-16S_v02.pdf}
	\caption[PCA ordination plots of bacterial taxa.]{
		PCA ordination plots (PC1, PC2) of the CLR-transformed relative abundances of bacterial taxa as classified by \textbf{a}) the 16S rDNA analysis or \textbf{b}) the metagenomic analysis in macrolide antibiotic-exposed and untreated control soil.
		PERMANOVA pseudo-$F$ and $p$-values with 999 permutations are displayed.
		Shaded areas correspond to 95\% confidence ellipses of treatment groups.
		Percentages of variance explained by each axis are displayed in the axis titles.
	}
	\label{supp-fig:pca-ordplots-taxa}
\end{suppfigure}

\begin{suppfigure}[H]
	\centering
		\includegraphics[width=\textwidth]{supp-figures/metagenomic/set-compartments-args-mges_v03.pdf}
	\caption[Number of antibiotic resistance genes and mobile genetic elements detected within each compartment formed between the untreated control, low-, and high-dosed soil groups.]{
		Number of antibiotic resistance genes and mobile genetic elements detected within each compartment formed between the untreated control, low-, and high-dosed soil groups, arranged by compartment size.
		Shaded dots below the bar plots correspond to the compartment.
	}
	\label{supp-fig:args-mges-compartment-size}
\end{suppfigure}

\begin{suppfigure}[H]
	\centering
		\includegraphics[width=0.75\textwidth]{supp-figures/integron/percentages-cog-categories_v02.pdf}
	\caption[Prevalence of COG functional categories assigned to integron gene cassette open reading frames.]{
		Prevalence (in percentages) of COG functional categories among integron gene cassette open reading frames that were assigned a COG function category (\textit{n} = 5,206).
		Only COG functional categories with a prevalence over 0.5\% are shown.
		Y-axis is logarithmically scaled and begins at 0.1\% for visual clarity.
		\todo[color=blue!20]{Is it okay for the y-axis to begin at 0.1\% instead of 0\%?}
	}
	\label{supp-fig:percentages-cog-categories}
\end{suppfigure}

%-------------------------------------------------------------------------------
% Supplementary Tables
%-------------------------------------------------------------------------------

\begin{supptable}[H]
	\centering
		\includegraphics[width=\textwidth]{supp-tables/primers_v01.pdf}
	\caption[PCR primer sequences, annealing temperatures, and expected amplicon sizes for integron gene cassette and 16S rDNA PCR amplification.]{
		PCR primer sequences, annealing temperatures, and expected amplicon sizes for integron gene cassette and 16S rDNA PCR amplification.
    Red text indicates location of adapter overhang sequences.
		\todo{Increase font size for this table.}
	}
	\label{supp-table:primers}
\end{supptable}

\begin{supptable}[H]
	\centering
		\includegraphics[width=\textwidth]{supp-tables/16S/16S-sequencing-statistics_v01.pdf}
	\caption[Summary of sequencing statistics for the 16S rDNA sequence dataset.]{
		Summary of sequencing statistics for the 16S rDNA sequence dataset.
		\todo{Increase font size for this table.}
	}
	\label{supp-table:16S-sequencing-statistics}
\end{supptable}

\end{appendices}

%CV only relevant stuff... not full CV.
\addcontentsline{toc}{chapter}{Curriculum Vitae}
\chapter*{Curriculum Vitae}
\begin{table}[ht]
\begin{tabular}{ll}
\textbf{Name:} & \firstname{} \lastname\\\\
\textbf{Post-Secondary} & La La School\\
\textbf{Education and}& La La Land\\
\textbf{Degrees:}& 1996 - 2000 M.A.\\\\
& University of Western Ontario\\
& London, ON\\
& 2008 - 2012 Ph.D.\\\\
\textbf{Honours and}& NSERC PGS M\\
\textbf{Awards:}& 2006-2007\\\\
\textbf{Related Work}& Teaching Assistant\\
\textbf{Experience:}& The University of Western Ontario\\
& 2008 - 2012\\
\end{tabular}
\end{table}
\subsubsection*{Publications:}
La La

\todo{Fill this out.}

\end{document}
