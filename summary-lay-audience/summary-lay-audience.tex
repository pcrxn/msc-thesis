\chapter*{Summary for Lay Audience}

Antibiotics --- chemicals that are used to treat bacterial infections --- are widespread pollutants in the environment due to their mass production and use in human and veterinary medicine.
Antibiotics that are consumed by people enter wastewater treatment plants where most treatment processes are unable to effectively eliminate these antibiotics.
The persistent antibiotics may then contaminate the environment and promote antibiotic resistance in environmental bacteria.

Macrolide antibiotics are a group of antibiotics that are frequently detected in wastewater due to their use in human medicine and are important to human health.
Unfortunately, bacterial pathogens (which are acquired from the environment) are becoming increasingly resistant to macrolide antibiotics.
Therefore, there is an urgent need to understand the effects of macrolide antibiotic pollution on environmental bacteria to reduce the increase of antibiotic resistance in human pathogens.

One pathway for macrolides to enter the environment and possibly promote antibiotic resistance is through the use of treated sewage sludge (biosolids) as agricultural fertilizer.
Macrolide antibiotics carry-over from wastewater into biosolids and then enter agricultural soil upon land-application.
Macrolide antibiotics are known to promote antibiotic resistance in soil bacteria when present at high levels, but their effect on soil bacteria at environmentally realistic levels remains unclear.

I investigated if an environmentally realistic dose of macrolides for a biosolids exposure event may promote antibiotic resistance in soil bacteria.
In 2010, 2 m\textsuperscript{2} soil field plots were established in London, ON and were annually exposed to a mixture of macrolides at an environmentally realistic dose or an effect-inducing unrealistically high dose, or were left antibiotic-free.
After ten years, DNA from each of these soil plots was isolated, sequenced, and then analyzed to investigate changes in the soil bacterial community that may have occurred due to antibiotic exposure.

I determined that an environmentally realistic dose of macrolides is unlikely to harm soil bacterial diversity or promote clinically relevant antibiotic resistance.
However, at an unrealistically high dose, genetic elements which encode and promote antibiotic resistance were increased.
Further studies should investigate intermediate macrolide doses which may pose a risk to human health by promoting antibiotic resistance in soil bacteria.
