Antimicrobial resistance (AMR) is the natural phenomenon by which microorganisms acquire defences against harmful chemicals within their environment.
All microorganisms have the potential to acquire AMR under various environments, including terrestrial and aquatic ecosystems, agri-food, hospitals, and the human body.
In the context of an infection, resistant microorganisms limit available treatment options.
Acquired antimicrobial resistance (AMR) is estimated to have caused 5,400 Canadian fatalities in 2015 — a number which is expected to climb to 13,700 deaths per year by 2050, while disproportionally affecting populations that are at a greater risk of acquiring infections \cite{councilofcanadianacademiesWhenAntibioticsFail2019}.

AMR is acquired in response to any chemical that imposes a selective pressure on the microorganism.
AMR is a huge burden to global healthcare systems, economies, and societies.
As \cite{buongerminopereiraComprehensiveSurveyIntegronassociated2020} has said...
