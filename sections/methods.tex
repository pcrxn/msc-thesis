\subsection{C1I gene cassette sequencing}

\subsubsection{DNA isolation from soil microplots}

Total genomic DNA was isolated from 2019 Day 30 post-application soil microplots using the DNeasy PowerSoil Kit (Qiagen) with an input of 250 mg of dry soil.
The DNA was eluted in 100 uL of 10 mM Tris-HCl, and Nanodrop readings were taken to assess quality.
The isolated DNA was stored at -20°C until it was used as template for gene cassette PCR.

\subsubsection{Gene cassette PCR}

Class 1 integron gene cassettes were amplified using primers and thermocycling conditions described by \textcite{stokesGeneCassettePCR2001}, except with 33 and 34 bp oligonucleotides ligated onto the 5’ ends \dummyfig.
The purpose of these oligonucleotides was to provide more distance between the 5’ end of the PCR product (lost during library preparation) and the desired gene cassette sequence.
DNA from each soil microplot was used as template for five technical replicates of 25 uL reactions (125 uL total).
PCR product was analyzed by gel electrophoresis and visualized under UV light.
Technical replicates were pooled together and PCR product was purified using the GenepHlow PCR Cleanup Kit (Geneaid).
Cleaned PCR product was quantified using the Qubit dsDNA HS Assay Kit (Fisher Scientific).
PCR and cleanup were repeated as described above using the same template DNA for a total of two technical replicates for each soil microplot.

\subsubsection{Library preparation}

The Nextera XT DNA Library Preparation Kit (Illumina) was used to prepare the library for sequencing by following the manufacturer’s protocol.
The indexed PCR product was quantified using the Qubit dsDNA HS Assay Kit and sized using the High Sensitivity DNA Kit on a Bioanalyzer 2100 (Agilent).
Individual libraries were diluted to 10 nM, and 15 uL of each diluted library was pooled into a single library for 2 x 125 bp sequencing at Sick Kid’s Hospital (Toronto, ON) on a HiSeq 2500 (Illumina).

\subsubsection{Data processing}

Adapters were trimmed using CUTADAPT (v2.10) and the sequence quality was assessed using FastQC (v0.11.9) and MultiQC (v1.8) \parencite{martinCutadaptRemovesAdapter2011, andrewsFastQCQualityControl2010, ewelsMultiQCSummarizeAnalysis2016}.
The adapter-trimmed reads were assembled into contigs using MEGAHIT (v1.2.9) with default options \parencite{liMEGAHITUltrafastSinglenode2015}.
Contigs that didn't contain the terminal 9 bp of each primer sequence were removed using BBDuk (options: \texttt{copyundefined=t rcomp=f mm=f k=9}) and the open reading frames (\textbf{ORFs}) of the remaining contigs were called using Prokka \parencite{bushnellBBMap2020, seemannProkkaRapidProkaryotic2014}.
The ORFs were concatenated into a non-redundant protein database at 97\% identity using CD-HIT (v\textcolor{red}{X.X}) and then reverse translated back to nucleotides \parencite{fuCDHITAcceleratedClustering2012}.
To obtain fold-coverages, the adapter-trimmed reads were mapped back onto the concatenated nucleotide database of ORFs using BBMap (v\textcolor{red}{X.X}) \parencite{bushnellBBMap2020}.
ORFs were annotated using CARD-RGI (v\textcolor{red}{X.X}, options: \texttt{--low\_quality --clean -t protein}) for ARGs and eggNOG-mapper (v2.0) for functional categories \parencite{alcockCARD2020Antibiotic2019, huerta-cepasFastGenomewideFunctional2017, huerta-cepasEggNOGHierarchicalFunctionally2019}.

\subsubsection{Compositional data analysis}
