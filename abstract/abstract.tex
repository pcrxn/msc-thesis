\Large\begin{center}\textbf{Abstract}\end{center}\normalsize

\noindent

Biosolids (treated sewage sludge) are used as agricultural fertilizer but are frequently contaminated with macrolide antibiotics, to which resistance is rising among historically susceptible bacteria.
To determine if the land-application of macrolides carried in biosolids could promote antibiotic resistance in soil bacteria, soil plots were exposed annually to environmentally realistic or high doses of macrolides for ten years.
I sequenced the bacterial 16S ribosomal DNA, metagenomic DNA, and integron gene cassettes within the treated and antibiotic-free soil to compare the compositions and diversities of the bacterial communities, antibiotic resistance genes, and mobile genetic elements.
I determined that the high dose treatment of macrolides, but not the realistic dose, increased the diversity of clinically relevant antibiotic resistance genes and mobile genetic elements and decreased the abundance of soil cyanobacteria.
Overall, typical concentrations of macrolides found in biosolids are unlikely to promote antibiotic resistance of concern to human health within soil bacteria.

\vfill

\noindent\textbf{Keywords:} agriculture, antibiotic resistance, bioinformatics, biosolids, ecotoxicology, integrons, metagenomics, microbial ecology, mobile genetic elements, soil
