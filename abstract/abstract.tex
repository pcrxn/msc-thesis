\addcontentsline{toc}{chapter}{Abstract}
\Large\begin{center}\textbf{Abstract}\end{center}\normalsize
%%  ***  Put your Abstract here.   ***
%% (150 words for M.Sc. and 350 words for Ph.D.)

\noindent
Biosolids are produced from treated wastewater and can be used as agricultural fertilizer. Due to the nature of their production, however, biosolids are frequently contaminated with macrolide antibiotics, to which drug resistance is rising among historically susceptible bacteria. To determine if the land-application of biosolids could increase clinically relevant antibiotic resistance in soil bacteria, we established soil plots and exposed them annually to an environmentally realistic or unrealistically high dose of macrolides for ten years. We sequenced the bacterial 16S ribosomal DNA, metagenomic DNA, and integron gene cassettes within the treated and untreated soil to compare the compositions and diversities of the bacterial communities, mobile genetic elements, and antibiotic resistance genes to that of antibiotic-free soil. We determined that the high dose but not the realistic dose of macrolides increased the diversity of clinically relevant antibiotic resistance genes and mobile genetic elements in soil and decreased the abundance of Cyanobacteria.

\vfill

\noindent\textbf{Keywords:} biosolids, antibiotic resistance, agriculture, soil, ecotoxicology, metagenomics.
