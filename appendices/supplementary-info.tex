\chapter{Supplementary Information}\label{AppA}
\myappendices{Appendix \ref{AppA}}

%-------------------------------------------------------------------------------
% Supplementary Figures
%-------------------------------------------------------------------------------

\newpage
\begin{suppfigure}[H]
	\centering
		\includegraphics[width=\textwidth]{supp-figures/16S/effect-sizes-sig-phyla-16S_v02.pdf}
	\caption[Effect sizes (fold-changes) of differences in the relative abundances of bacterial phyla classified in the 16S rDNA analysis relative to the untreated control soil (\textit{n} = 4 for antibiotic-exposed groups, \textit{n} = 3 for untreated control group).]{
		Effect sizes (fold-changes) of differences in the relative abundances of bacterial phyla classified in the 16S rDNA analysis relative to the untreated control soil (\textit{n} = 4 for antibiotic-exposed groups, \textit{n} = 3 for untreated control group).
		Horizontal lines intersecting with circles are error bars, indicating the extent of Bonferroni-adjusted 95\% confidence intervals of effect sizes.
	}
	\label{supp-fig:effect-sizes-sig-phyla-16S}
\end{suppfigure}

\begin{suppfigure}[H]
	\centering
		\includegraphics[width=0.8\textwidth]{supp-figures/mixed/chao1-richness-taxa_v02.pdf}
	\caption[Richness (Chao1) of bacterial taxa as classified by \textbf{a}) the 16S rDNA analysis or \textbf{b}) the metagenomic analysis in macrolide antibiotic-exposed and untreated control soil.]{
		Richness (Chao1) of bacterial taxa as classified by \textbf{a}) the 16S rDNA analysis or \textbf{b}) the metagenomic analysis in macrolide antibiotic-exposed and untreated control soil.
		There were no statistically significant differences between the antibiotic-exposed and -unexposed groups.
		\textsuperscript{\Cross} \textit{n} = 4 for the antibiotic-exposed groups, \textit{n} = 3 for the untreated control group.
	}
	\label{supp-fig:chao1-richness-taxa}
\end{suppfigure}

\begin{suppfigure}[htpb]
	\centering
		\includegraphics[width=0.75\textwidth]{supp-figures/mixed/pca-ordplots-taxa-metagenome-16S_v02.pdf}
	\caption[PCA ordination plots (PC1, PC2) of the CLR-transformed relative abundances of bacterial taxa as classified by \textbf{a}) the 16S rDNA analysis or \textbf{b}) the metagenomic analysis in macrolide antibiotic-exposed and untreated control soil.]{
		PCA ordination plots (PC1, PC2) of the CLR-transformed relative abundances of bacterial taxa as classified by \textbf{a}) the 16S rDNA analysis or \textbf{b}) the metagenomic analysis in macrolide antibiotic-exposed and untreated control soil.
		PERMANOVA pseudo-$F$ and $p$-values with 999 permutations are displayed.
		Shaded areas correspond to 95\% confidence ellipses of treatment groups.
		Percentages of variance explained by each axis are displayed in the axis titles.
	}
	\label{supp-fig:pca-ordplots-taxa}
\end{suppfigure}

\begin{suppfigure}[H]
	\centering
		\includegraphics[width=\textwidth]{supp-figures/metagenomic/set-compartments-args-mges_v03.pdf}
	\caption[Number of antibiotic resistance genes and mobile genetic elements detected within each compartment formed between the untreated control, low-, and high-dosed soil groups, arranged by compartment size.]{
		Number of antibiotic resistance genes and mobile genetic elements detected within each compartment formed between the untreated control, low-, and high-dosed soil groups, arranged by compartment size.
		Shaded dots below the bar plots correspond to the compartment.
	}
	\label{supp-fig:args-mges-compartment-size}
\end{suppfigure}

\begin{suppfigure}[H]
	\centering
		\includegraphics[width=0.75\textwidth]{supp-figures/integron/percentages-cog-categories_v02.pdf}
	\caption[Prevalence (in percentages) of COG functional categories among integron gene cassette open reading frames that were assigned a COG function category (\textit{n} = 5,206).]{
		Prevalence (in percentages) of COG functional categories among integron gene cassette open reading frames that were assigned a COG function category (\textit{n} = 5,206).
		Only COG functional categories with a prevalence over 0.5\% are shown.
		Y-axis is logarithmically scaled and begins at 0.1\% for visual clarity.
	}
	\label{supp-fig:percentages-cog-categories}
\end{suppfigure}

%-------------------------------------------------------------------------------
% Supplementary Tables
%-------------------------------------------------------------------------------

\begin{supptable}[H]
	\centering
		\includegraphics[width=\textwidth]{supp-tables/primers_v03.pdf}
	\caption[PCR primer sequences, annealing temperatures, and expected amplicon sizes for integron gene cassette and 16S rDNA PCR amplification \parencite{Stokes.2001, Klindworth.2013}.]{
		PCR primer sequences, annealing temperatures, and expected amplicon sizes for integron gene cassette and 16S rDNA PCR amplification \parencite{Stokes.2001, Klindworth.2013}.
		Degenerate bases follow the IUPAC standard ambiguity code.
    Red text indicates location of adapter overhang sequences.
	}
	\label{supp-table:primers}
\end{supptable}

\begin{supptable}[H]
	\centering
		\includegraphics[width=\textwidth]{supp-tables/16S/sequencing-statistics-16S_v02.pdf}
	\caption[16S rDNA amplicon sample sequence statistics for unprocessed (raw) and trimmed reads, and the percentages of input sequence reads merged by DADA2, with samples grouped by macrolide antibiotic dose.]{
	16S rDNA amplicon sample sequence statistics for unprocessed (raw) and trimmed reads, and the percentages of input sequence reads merged by DADA2, with samples grouped by macrolide antibiotic dose.
	Mean values are reported along with standard deviations (SD).
	}
	\label{supp-table:sequencing-statistics-16S}
\end{supptable}

\begin{supptable}[H]
	\centering
		\includegraphics[width=\textwidth]{supp-tables/metagenomic/sequencing-statistics-metagenomic_v01.pdf}
	\caption[Metagenomic DNA sample sequence statistics for unprocessed (raw) and trimmed reads, with samples grouped by macrolide antibiotic dose.]{
	Metagenomic DNA sample sequence statistics for unprocessed (raw) and trimmed reads, with samples grouped by macrolide antibiotic dose.
	The number of biological replicates within each treatment group are shown beside each analysis step.
	Mean values are reported along with standard deviations (SD).
	}
	\label{supp-table:sequencing-statistics-metagenomic}
\end{supptable}

\begin{supptable}[H]
	\centering
		\includegraphics[width=\textwidth]{supp-tables/integron/sequencing-statistics-integron_v02.pdf}
	\caption[Integron gene cassette amplicon sample sequence statistics for unprocessed (raw) reads, and assembly statistics for unfiltered (post-assembly) and filtered (post-cassette filtering) contigs, with samples grouped by macrolide antibiotic dose.]{
	Integron gene cassette amplicon sample sequence statistics for unprocessed (raw) reads, and assembly statistics for unfiltered (post-assembly) and filtered (post-cassette filtering) contigs, with samples grouped by macrolide antibiotic dose.
	The number of biological replicates within each treatment group are shown beside each analysis step.
	Mean values are reported along with standard deviations (SD).
	}
	\label{supp-table:sequencing-statistics-integron}
\end{supptable}

\begin{supptable}[H]
	\centering
		\includegraphics[width=0.70\textwidth]{supp-tables/integron/integron-args-drug-classes_v01.pdf}
	\caption[Counts of target drug classes for predicted antibiotic resistance genes within integron gene cassettes at each confidence level.]{
	Counts of target drug classes for predicted antibiotic resistance genes within integron gene cassettes at each confidence level.
	Values in the table are shaded based upon the minimum and maximum values in each column.
	}
	\label{supp-table:integron-args-drug-classes}
\end{supptable}
